\documentclass[a4paper,12pt]{article}

\usepackage[utf8]{inputenc}
\usepackage[left=0.5in,right=0.5in,top=0.75in,bottom=1in]{geometry}
\usepackage{amsmath,amssymb,amsfonts}
\usepackage{pgfplots,graphicx,calc,changepage}
\pgfplotsset{compat=newest}
\usepackage{enumitem}
\usepackage{fancyhdr}
\usepackage[colorlinks = true, linkcolor = blue]{hyperref}

% Syntax highlighting
\usepackage{listings}
\usepackage{xcolor}

\definecolor{codegreen}{rgb}{0.40,0.62,0.07}
\definecolor{codegray}{rgb}{0.5,0.5,0.5}
\definecolor{codeblue}{rgb}{0.09,0.57,0.73}
\definecolor{backcolour}{rgb}{1,1,1}

\lstdefinestyle{mystyle}{
    backgroundcolor=\color{backcolour},   
    commentstyle=\color{codegreen},
    keywordstyle=\color{magenta},
    numberstyle=\tiny\color{codegray},
    stringstyle=\color{codeblue},
    basicstyle=\ttfamily\small,
    breaklines=true,                     
    keepspaces=true,                 
    numbers=left,                    
    numbersep=5pt,                  
    showspaces=false,
    showstringspaces=false,
    showtabs=false,                  
    tabsize=4
}

\lstset{style=mystyle}

\lstdefinelanguage{Julia}%
{morekeywords={abstract,break,case,catch,const,continue,do,else,elseif,%
		end,export,false,for,function,immutable,import,importall,if,in,%
		macro,module,otherwise,quote,return,switch,true,try,type,typealias,%
		using,while},%
	sensitive=true,%
	alsoother={$},%
	morecomment=[l]\#,%
	morecomment=[n]{\#=}{=\#},%
	morestring=[s]{"}{"},%
	morestring=[m]{'}{'},%
}[keywords,comments,strings]%

\newcommand{\nats}{\mathbb{N}}
\newcommand{\reals}{\mathbb{R}}
\newcommand{\rats}{\mathbb{Q}}
\newcommand{\ints}{\mathbb{Z}}
\newcommand{\comps}{\mathbb{C}}
\newcommand{\pols}{\mathcal{P}}
\newcommand{\cants}{\Delta\!\!\!\!\Delta}
\newcommand{\eps}{\varepsilon}
\newcommand{\st}{\backepsilon}
\newcommand{\abs}[1]{\left| #1 \right|}
\newcommand{\dom}[1]{\mathrm{dom}\left(#1\right)}
\newcommand{\for}{\text{ for }}
\newcommand{\dd}{\mathrm{d}}
\newcommand{\spn}{\mathrm{sp}}
\newcommand{\nul}{\mathcal{N}}
\newcommand{\col}{\mathrm{col}}
\newcommand{\rank}{\mathrm{rank}}
\newcommand{\norm}[1]{\lVert #1 \rVert}
\newcommand{\inner}[1]{\left\langle #1 \right\rangle}
\newcommand{\pmat}[1]{\begin{pmatrix} #1 \end{pmatrix}}
\renewcommand{\and}{\text{ and }}

\newsavebox{\qed}
\newenvironment{proof}[2][$\square$]
    {\setlength{\parskip}{0pt}\par\textit{Proof:} #2\setlength{\parskip}{0.25cm}
        \savebox{\qed}{#1}
        \begin{adjustwidth}{\widthof{Proof:}}{}
    }
    {
        \hfill\usebox{\qed}\end{adjustwidth}
    }

\pagestyle{fancy}
\fancyhead{}
\lhead{Caleb Jacobs}
\chead{APPM 5610: Numerical Analysis II}
\rhead{Homework \#9}
\cfoot{}
\setlength{\headheight}{35pt}
\setlength{\parskip}{0.25cm}
\setlength{\parindent}{0pt}

\begin{document}
\begin{enumerate}[label = (\arabic*)]
	\item Implement the Crank-Nicolson scheme for the heat equation
	\[
		\begin{cases}
			u_t = \partial_x (a(x) u_x) + f(x,t), & t > 0, x\in (0,1) \\
			u(x,0) = u_0(x), & x \in [0,1] \\
			u(0,t) = u(1,t) = 0, & t > 0
		\end{cases}
	\]
	
	To implement Crank-Nicolson, we need to find our $ F(x,t,u,u_x, u_{xx}) $ operator. For this PDE, we simply have
	\[
		F(x,t,u,u_x,u_{xx}) = \partial_x(a(x) u_x) + f(x,t) = a(x) u_{xx} + a'(x) u_x + f(x,t).
	\]
	Then, our Crank-Nicolson scheme is given by
	\[
		\frac{u_i^{n + 1} - u_i^n}{h_t} = \frac{1}{2}\left(F_i^{n + 1}(u, x, t, u_x, u_{xx}) + F_i^{n}(u, x, t, u_x, u_{xx})\right)
	\]
	where $ F_i^n $ represents the second order finite difference version of $ F $ (I use central differences). Plugging in the finite differences, we have
	\[
		F_i^n = a(x_i) \frac{u_{i+1}^n - 2u_{i}^n + u_{i - 1}^n}{h_x^2} + a'(x_i) \frac{u_{i+1}^n - u_{i - 1}^n}{2h_x} + f(x_i, t_n).
	\]
	So, taking this expression for $ F_i^n $ and it plugging into our Crank-Nicolson scheme yields the linear tridiagonal system that I wrote my code to solve (Code attached at the end of the document).

	To check that my code is working, I ran it on the test cases below:
	\begin{enumerate}[label = (\alph*)]
		\item Standard heat equation
			\[
				\begin{cases}
					u_t = u_{xx}, & t > 0, x\in (0,1) \\
					u(x,0) = -4x(x-1), & x \in [0,1] \\
					u(0,t) = u(1,t) = 0, & t > 0
				\end{cases}
			\]
			
		\item Simple forced heat equation
			\[
				\begin{cases}
					u_t = u_{xx} + x^2 t, & t > 0, x\in (0,1) \\
					u(x,0) = -4x(x-1), & x \in [0,1] \\
					u(0,t) = u(1,t) = 0, & t > 0
				\end{cases}
			\]
		
		\item Spatially variable conductivity
			\[
				\begin{cases}
					u_t = \partial_x(x u_x) & t > 0, x\in (0,1) \\
					u(x,0) = -4x(x-1), & x \in [0,1] \\
					u(0,t) = u(1,t) = 0, & t > 0
				\end{cases}
			\]
			
		\item Ill-posed heat equation
			\[
				\begin{cases}
					u_t = -u_{xx} & t > 0, x\in (0,1) \\
					u(x,0) = -4x(x-1), & x \in [0,1] \\
					u(0,t) = u(1,t) = 0, & t > 0
				\end{cases}
			\]
	\end{enumerate}
	In every case except case $ d $, my code ran stably and accurately for many different spatial and temporal step sizes. However, for case (d), my code leads to an ``exploding'' solution which is expected because the problem doesn't have continuous dependence on the initial data and is thus ill-posed.
		
	\newpage
	\item Implement the second-order central difference scheme for the wave equation
	\[
		\begin{cases}
			u_{tt} = \partial_x (a(x) u_x) + f(x,t), \\
			u(x,0) = u_0(x), \\
			u_t(x,0) = u_1(x)
		\end{cases}
	\]
	where all functions are periodic in $ x $ with period 1.
	
	To turn this into a finite difference problem, let's first expand the PDE as
	\[
		u_{tt} = a(x) u_{xx} + a'(x) u_x + f(x,t).
	\]
	Then, plugging in our central differences, we have
	\[
		\frac{u_i^{n + 1} - 2u_i^n + u_i^{n - 1}}{h_t^2} = a(x_i) \frac{u_{i+1}^n - 2u_{i}^n + u_{i - 1}^n}{h_x^2} + a'(x_i) \frac{u_{i+1}^n - u_{i - 1}^n}{2h_x} + f(x_i, t_n)
	\]
	which yields the explicit time stepping scheme
	\[
		u_i^{n + 1} = (r + k) u_{i + 1}^n + (2 - 2r)u_i^n + (r - k)u_{i - 1}^n - u_i^{n - 1} + h_t^2 f(x_i, t_n)
	\]
	where
	\[
		r = a(x_i) \frac{h_t^2}{h_x^2} \quad\and\quad k = a'(x_i) \frac{h_t^2}{2h_x}.
	\]
	Now, this scheme works great on the interior for $ t > 0 $. However, at the very first step, we aren't directly given the $ u_i^{n - 1} $ data and so we need to rely on our initial data. In this case, we can use the condition $ u_t(x, 0) = u_1(x) $ along with finite differences to get
	\[
		\frac{u_i^2 - u_i^0}{2h_t} = u_1(x_i) \implies u_i^0 = u_i^2 - 2h_t u_1(x_i).
	\]
	This expression can then be plugged into our explicit scheme to yield a slightly modified time stepping scheme for the first time step. My code at the end of the document implements this exact scheme. Do note, I added periodic boundary conditions to my code to make it well posed over the interval for $ x $ from 0 to 1.
	
	I ran my code on the test cases below:
	\begin{enumerate}[label = (\alph*)]
		\item Standard wave equation
			\[
				\begin{cases}
					u_{tt} = u_{xx}, \\
					u(x,0) = \sin(2\pi x), \\
					u_t(x,0) = 0
				\end{cases}
			\]
			
		\item Resonant wave equation
			\[
				\begin{cases}
					u_{tt} = u_{xx} + \sin(2\pi x) \cos(2\pi t), \\
					u(x,0) = 0, \\
					u_t(x,0) = 0
				\end{cases}
			\]
			
		\item Plucked string
			\[
				\begin{cases}
					u_{tt} = u_{xx}, \\
					u(x,0) = 0, \\
					u_t(x,0) = -4x(x - 1)
				\end{cases}
			\]
			
		\item Variable tension string
			\[
				\begin{cases}
					u_{tt} = \partial_x(x u_x), \\
					u(x,0) = \sin(2\pi x), \\
					u_t(x,0) = 0
				\end{cases}
			\]
		
		\item Ill-posed wave equation
			\[
				\begin{cases}
					u_{tt} = -u_{xx}, \\
					u(x,0) = \sin(2\pi x), \\
					u_t(x,0) = 0
				\end{cases}
			\]
	\end{enumerate}
	
	In all cases, aside from case (e), my code runs stably so long as we have $ r $ and $ k $ defined above less than 1 which gives us constraints on the relative sizes of $ h_x $ and $ h_t $. So, as long as we pick $ h_x $ and $ h_t $ such that $ \abs{r} < 1 $ and, $ \abs{k} < 1 $, our scheme will be stable. For accuracy however, we desire a smaller $ h_t $ as well. 
	
	For case (e), our scheme leads to many growing amplitude, fast oscillation solutions which comes from the ill-posedness of the problem similar to the ill-posed heat equation.
\end{enumerate}

\newpage
\section*{Code Used}

\emph{Note: some of the symbols are missing in my code snippet because \LaTeX\ does not support all unicode characters.}

\textbf{Crank-Nicolson Code}

\vspace{-0.5cm}
\rule{\textwidth}{.4pt}
\lstinputlisting[language = Julia]{Code/Crank_Nicolson.jl}
\rule{\textwidth}{.4pt}

\newpage
\textbf{Central Difference Wave Code}

\vspace{-0.5cm}
\rule{\textwidth}{.4pt}
\lstinputlisting[language = Julia]{Code/Wave_FD.jl}
\rule{\textwidth}{.4pt}

\end{document}