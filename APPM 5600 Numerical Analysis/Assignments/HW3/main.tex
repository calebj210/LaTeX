\documentclass[a4paper,12pt]{article}

\usepackage[utf8]{inputenc}
\usepackage[left=0.5in,right=0.5in,top=1in,bottom=1in]{geometry}
\usepackage{amsmath,amssymb,amsfonts}
\usepackage{pgfplots,graphicx,calc,changepage}
\pgfplotsset{compat=newest}
\usepackage{enumitem}
\usepackage{fancyhdr}
\usepackage[colorlinks = true, linkcolor = blue]{hyperref}

% Syntax highlighting
\usepackage{listings}
\usepackage{xcolor}

\definecolor{codegreen}{rgb}{0.40,0.62,0.07}
\definecolor{codegray}{rgb}{0.5,0.5,0.5}
\definecolor{codeblue}{rgb}{0.09,0.57,0.73}
\definecolor{backcolour}{rgb}{1,1,1}

\lstdefinestyle{mystyle}{
    backgroundcolor=\color{backcolour},   
    commentstyle=\color{codegreen},
    keywordstyle=\color{magenta},
    numberstyle=\tiny\color{codegray},
    stringstyle=\color{codeblue},
    basicstyle=\ttfamily\small,
    breaklines=true,                     
    keepspaces=true,                 
    numbers=left,                    
    numbersep=5pt,                  
    showspaces=false,
    showstringspaces=false,
    showtabs=false,                  
    tabsize=4
}

\lstset{style=mystyle}

\newcommand{\nats}{\mathbb{N}}
\newcommand{\reals}{\mathbb{R}}
\newcommand{\rats}{\mathbb{Q}}
\newcommand{\ints}{\mathbb{Z}}
\newcommand{\pols}{\mathcal{P}}
\newcommand{\cants}{\Delta\!\!\!\!\Delta}
\newcommand{\eps}{\varepsilon}
\newcommand{\st}{\backepsilon}
\newcommand{\abs}[1]{\left| #1 \right|}
\newcommand{\dom}[1]{\mathrm{dom}\left(#1\right)}
\newcommand{\for}{\text{ for }}
\newcommand{\dd}[1]{\mathrm{d}#1}
\newcommand{\spn}{\mathrm{sp}}
\newcommand{\nul}{\mathcal{N}}
\newcommand{\col}{\mathrm{col}}
\newcommand{\rank}{\mathrm{rank}}
\newcommand{\norm}[1]{\lVert #1 \rVert}
\newcommand{\inner}[1]{\left\langle #1 \right\rangle}
\newcommand{\pmat}[1]{\begin{pmatrix} #1 \end{pmatrix}}
\renewcommand{\and}{\text{ and }}

\newsavebox{\qed}
\newenvironment{proof}[2][$\square$]
    {\setlength{\parskip}{0pt}\par\textit{Proof:} #2\setlength{\parskip}{0.25cm}
        \savebox{\qed}{#1}
        \begin{adjustwidth}{\widthof{Proof:}}{}
    }
    {
        \hfill\usebox{\qed}\end{adjustwidth}
    }

\pagestyle{fancy}
\fancyhead{}
\lhead{Caleb Jacobs}
\chead{APPM 5600: Numerical Analysis I}
\rhead{Homework \#3}
\cfoot{}
\setlength{\headheight}{35pt}
\setlength{\parskip}{0.25cm}
\setlength{\parindent}{0pt}

\begin{document}
\begin{enumerate}[label = \arabic*.]
	\item Let $ x_0, x_1 $ be two successive points from a secant method applied to solving $ f(x) = 0 $ with $ f_0 = f(x_0), f_1 = f(x_1) $. Show that regardless of which point $ x_0 $ or $ x_1 $ is regarded as the most recent point, the new point derived from the secant step will be the same.
	
	Suppose $ x_1 $ is the most recent point. Then the next point $ x_2 $ produced by the secant method would be given by
	\begin{align*}
		x_2 &= x_1 - f_1 \frac{x_1 - x_0}{f_1 - f_0} \\
		&= \frac{x_1(f_1 - f_0) - f_1(x_1 - x_0)}{f_1 - f_0} \\
		&= \frac{x_1 f_1 - x_1 f_0 - x_1 f_1 + x_0 f_1}{f_1 - f_ 0} \\
		&= \frac{x_0 f_1 - x_1 f_0}{f_1 - f_0} \\
		&= \frac{x_0 f_1 - x_1 f_0 + x_0 f_0 - x_0 f_0}{f_1 - f_0} \\
		&= \frac{x_0 (f_1 - f_0) + f_0(x_0 - x_1) }{f_1 - f_0} \\
		&= x_0 - f_0 \frac{x_0 - x_1}{f_0 - f_1}
	\end{align*}
	which is the secant iteration if $ x_0 $ was the most recent point. Thus, regardless of which point is the most recent, the secant iteration will produce the same point $ x_2 $.
	
	\item Determine whether the following sets of vectors are dependent or linearly independent:
	\begin{enumerate}[label = (\alph*)]
		\item $ (1,2,-1,3), (3,-1,1,1), (1,9,-5,11) $.
		
		We can determine if this set of vectors in linearly independent by forming the matrix and row reducing as follows
		\[
			\begin{pmatrix}
				 1 &  3 &  1 \\
				 2 & -1 &  9 \\
				-1 &  1 & -5 \\
				 3 &  1 & 11
			\end{pmatrix}
			\sim
			\begin{pmatrix}
				1 & 0 & 4 \\
				0 & 1 & -1 \\
				0 & 0 & 0 \\
				0 & 0 & 0
			\end{pmatrix}.
		\]
		This row reduction shows that our third vector can be written as a linear combination of the first two vectors and so our set is \emph{linearly dependent}.
		
		\item $ (1,1,0), (0,1,1), (1,0,1) $.
		
		Just as in part (a), we can test linear dependence by forming a matrix and row reducing as follows
		\[
			\begin{pmatrix}
				1 & 0 & 1 \\
				1 & 1 & 0 \\
				0 & 1 & 1
			\end{pmatrix}
			\sim
			\begin{pmatrix}
				1 & 0 & 0 \\
				0 & 1 & 0 \\
				0 & 0 & 1
			\end{pmatrix}.
		\]
		The row reduction shows that each vector can not be written as a linear combination of the other two vectors and so our set is \emph{linearly independent}.
	\end{enumerate}

	\newpage
	\item Let $ \vec{x}_1, \vec{x}_2, \ldots, \vec{x}_k $ be linearly independent vectors in $ \reals^n $ and let $ A $ be a non-singular $ n \times n $ matrix. Define $ \vec{y}_i = A \vec{x}_i $ for $ i = 1, 2, \ldots, k $. Show that $ \vec{y}_1, \vec{y}_2, \ldots, \vec{y}_k $ are linearly independent.
	
	\begin{proof}{}
		Let $A, x_i, $ and $ y_i $ be defined as above for $ i = 1, 2, \ldots, k $. Then, because $ \{x_i\}_{i=1}^k $ forms a linearly independent set, $ x_i \neq 0 $ for $ i = 1,2, \ldots, k $. Furthermore, because $ A $ is non-singular and $ x_i \neq 0 $,
		\[
			y_i = A x_i \neq 0
		\]
		for each $ i = 1,2, \cdots, k $. With this information in mind, let's find constants $ a_i $ such that $ a_1 y_1 + a_2 y_2 + \cdots + a_k y_k = 0 $. Consider
		\begin{align*}
			a_1 y_1 + a_2 y_2 + \cdots + a_k y_k &= a_1 A x_1 + a_2 A x_2 + \cdots + a_k A x_k \\
			&= A (a_1 x_1 + a_2 x_2 + \cdots + a_k x_k). \label{equ:lincomb}\tag{$*$}
		\end{align*}
		Then, because $ A $ is non-singular, setting \eqref{equ:lincomb} equal to zero yields
		\begin{equation}
			a_1 x_1 + a_2 x_2 + \cdots + a_k x_k = 0. \label{equ:linindep}
		\end{equation}
		Finally, because $ \{x_i\}_{i=1}^k $ is linearly independent, the only solution to \eqref{equ:linindep} is $ a_i = 0 $ for each $ i = 1, 2, \ldots, k $. Thus, the only solution to $ a_1 y_1 + a_2 y_2 + \cdots + a_k y_k = 0 $ is when $ a_i = 0 $ for each $ i = 1, 2, \ldots, k $. Therefore, $ \{y_i\}_{i=1}^k $ forms a linearly independent set.
	\end{proof}

	\item Given the orthogonal vectors
	\[
		\vec{u}_1 = (1,2,-1) \quad \vec{u}_2 = (1, 1, 3)
	\]
	produce a third vector $ \vec{u}_3 $ such that $ \{\vec{u}_1, \vec{u}_2, \vec{u}_3\} $ is an orthogonal basis for $ \reals^3 $. Normalize the vectors to create an orthonormal basis.
	
	For any number of dimensions, we could use the Gram-Schmidt process to generate orthogonal vectors to our given vectors but we are in $ \reals^3 $ and so we can simply use the cross product to get a third orthogonal vector $ \vec{u}_3 $. We can produce $ \vec{u}_3 $ as
	\[
		\vec{u}_3 = \vec{u}_1 \times \vec{u}_2
		=
		\begin{vmatrix}
			\hat{i} & \hat{j} & \hat{k} \\
			1 & 2 & -1 \\
			1 & 1 & 3
		\end{vmatrix}
		=
		(6 + 1)\hat{i} - (3 + 1)\hat{j} + (1 - 2)\hat{k} = (7, -4, -1)
	\]
	Thus, $ \{\vec{u}_1, \vec{u}_2, \vec{u}_3\} = \{(1,2,-1), (1,1,3), (7,-4,-1)\} $ is an orthogonal basis for $ \reals^3 $. Going further, we can normalize each vector to get the orthonormal basis
	\[
		\left\{
		\frac{1}{\sqrt{6}} \pmat{1 \\ 2 \\ -1},
		\frac{1}{\sqrt{11}} \pmat{1 \\ 1 \\ 3},
		\frac{1}{\sqrt{66}} \pmat{7 \\ -4 \\ -1} 
		\right\}.
	\]
	
	\item Prove that similar matrices have the same eigenvalues and that there is a one-to-one correspondence of the eigenvectors.
	
	\begin{proof}{}
		Suppose we have similar square $ n \times n $ matrices $ A $ and $ B $. Then, by the definition of similar matrices, there exists an invertible $ n \times n $ matrix $ P $ such that
		\begin{equation}
			A = P^{-1} B P. \label{equ:sim}
		\end{equation}
		Now, suppose $ A $ has an eigenvalue $ \lambda $ with corresponding eigenvector $ \vec{\lambda} $. Then, $ A \vec{\lambda} = \lambda \vec{\lambda} $. Furthermore,
		\[
			P^{-1} B P \vec{\lambda} = A \vec{\lambda} = \lambda \vec{\lambda}.
		\]
		Rearranging yields
		\begin{equation}
			B (P \vec{\lambda}) = \lambda (P \vec{\lambda}) \label{equ:eig}
		\end{equation}
		which implies $ P \vec{\lambda} $ is an eigenvector of $ B $ with corresponding eigenvalue $ \lambda $. Thus, $ A $ and $ B $ both have the same eigenvalue and because we picked any eigenvalue of $ A $ and $ A $ and $ B $ are the same size, $ A $ and $ B $ must have the same eigenvalues. Furthermore, from \eqref{equ:eig}, we can form a one-to-one correspondence between the eigenvectors $ \vec{\lambda}_A $ of $ A $ to the eigenvectors $ \vec{\lambda}_B $ of $ B $ as
		\[
			\vec{\lambda}_B = P \vec{\lambda}_A
		\]
		because $ P $ is invertible and thus one-to-one in mapping eigenvectors of $ A $ to of eigenvectors $ B $.
	\end{proof}

    \item A matrix $ A \in \reals^{n \times n} $ is \emph{positive definite} if and only if $ \inner{A \vec{x}, \vec{x}} > 0 $ for all $ \vec{x} \in \reals^n; x \neq 0 $,
    
    Prove that if $ A $ is positive definite, then $ A $ is non-singular.
    
    \begin{proof}{Proof by contradiction:}
        Suppose $ A \in \reals^{n \times n} $ is a positive definite matrix. Then, $ \inner{A \vec{x}, \vec{x}} > 0 $ for all non-zero $ x \in \reals^n $. 
        
        Now, suppose there exists some $ \vec{x} \neq 0 $ such that $ A\vec{x} = 0 $. Then
        \[
            \inner{A\vec{x}, \vec{x}} = \inner{0, \vec{x}} = 0.
        \]
        But, because $ A $ is positive definite, we know if $ \vec{x} \neq 0 $, then $ \inner{A\vec{x}, \vec{x}} > 0$ which is a contradiction. Therefore, $ A\vec{x} = 0 $ only if $ \vec{x} = 0 $ which shows that $ A $ is non-singular.
    \end{proof}

    \item Let $ M $ be any real $ n \times n $ non-singular matrix and let $ A = M^T M $. Prove that $ A $ is positive definite.
    
    \begin{proof}{}
        Let $ M $ and $ A $ be defined as above and let any non-zero $ \vec{x} \in \reals^n $ be given. Then
        \begin{align*}
            \vec{x}^T A \vec{x} &= \vec{x}^T M^T M \vec{x} \\
            &= (M \vec{x})^T (M \vec{x}) \\
            &= (M \vec{x}) \cdot (M \vec{x}).
        \end{align*}
        Now, we have the dot product between two identical vectors. Furthermore, because $ M $ is non-singular and $ \vec{x} \neq 0 $, we must have $ M \vec{x} \neq 0 $. Finally, because the dot product is an inner product, the dot product between two identical non-zero vectors must be strictly positive. So, putting everything together, we must have
        \begin{equation}
            \vec{x}^T A \vec{x} = (M \vec{x}) \cdot (M \vec{x}) > 0. \label{equ:posdef}
        \end{equation}
        Then, because the choice of non-zero $ \vec{x} $ was arbitrary, \eqref{equ:posdef} must hold for all non-zero $ x \in \reals^n $ which by definition means $ A $ is positive definite.
    \end{proof}
\end{enumerate}
\end{document}