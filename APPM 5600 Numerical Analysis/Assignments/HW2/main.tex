\documentclass[a4paper,12pt]{article}

\usepackage[utf8]{inputenc}
\usepackage[left=0.5in,right=0.5in,top=1in,bottom=1in]{geometry}
\usepackage{amsmath,amssymb,amsfonts}
\usepackage{pgfplots,graphicx,calc,changepage}
\pgfplotsset{compat=newest}
\usepackage{enumitem}
\usepackage{fancyhdr}
\usepackage[colorlinks = true, linkcolor = blue]{hyperref}

\newcommand{\nats}{\mathbb{N}}
\newcommand{\reals}{\mathbb{R}}
\newcommand{\rats}{\mathbb{Q}}
\newcommand{\ints}{\mathbb{Z}}
\newcommand{\pols}{\mathcal{P}}
\newcommand{\cants}{\Delta\!\!\!\!\Delta}
\newcommand{\eps}{\varepsilon}
\newcommand{\st}{\backepsilon}
\newcommand{\abs}[1]{\left| #1 \right|}
\newcommand{\dom}[1]{\mathrm{dom}\left(#1\right)}
\newcommand{\erf}{\mathrm{erf}}
\newcommand{\for}{\text{ for }}
\newcommand{\dd}{\mathrm{d}}
\newcommand{\spn}{\mathrm{sp}}
\newcommand{\nul}{\mathcal{N}}
\newcommand{\col}{\mathrm{col}}
\newcommand{\rank}{\mathrm{rank}}
\newcommand{\norm}[1]{\lVert #1 \rVert}
\newcommand{\inner}[1]{\left\langle #1 \right\rangle}
\newcommand{\pmat}[1]{\begin{pmatrix} #1 \end{pmatrix}}
\renewcommand{\and}{\text{ and }}

\newsavebox{\qed}
\newenvironment{proof}[2][$\square$]
{\setlength{\parskip}{0pt}\par\textit{Proof:} #2\setlength{\parskip}{0.25cm}
	\savebox{\qed}{#1}
	\begin{adjustwidth}{\widthof{Proof:}}{}
	}
	{
		\hfill\usebox{\qed}\end{adjustwidth}
}

\pagestyle{fancy}
\fancyhead{}
\lhead{Caleb Jacobs}
\chead{APPM 5600: Numerical Analysis I}
\rhead{Homework \#2}
\cfoot{}
\setlength{\headheight}{35pt}
\setlength{\parskip}{0.25cm}
\setlength{\parindent}{0pt}

\begin{document}
	\begin{enumerate}[label = \arabic*.]
		\item Which of the following iterations will converge to the indicated fixed point $ x_* $ (provided $ x_0 $ is sufficiently close to $ x_* $)? If it does converge, give the order of convergence; for linear convergence, give the rate of linear convergence.
		
		\begin{enumerate}[label = \roman*.]
			\item $ x_{n+1} = -16 + 6x_n + \frac{12}{x_n}, x_*= 2$
			
			\item $ x_{n+1} = \frac{2}{3} x_n + \frac{1}{x_n^2}, x_* = 3^{1/3}$
			
			\item $ x_{n+1} = \frac{12}{1+x_n}, x_* = 3 $
		\end{enumerate}
	
		\item In laying water mains, utilities must be concerned with the possibility of freezing. Although soil and weather conditions are complicated, reasonable approximations can be made on the basis of the assumption that soil is uniform in all directions.  In that case the temperature in degrees Celsius $ T(x,t) $ at a distance $ x $ (in meters) below the surface, $ t $ seconds after the beginning of a cold snap, approximately satisfies
		\[
			\frac{T(x,t) - T_s}{T_i - T_s} = \erf\left(\frac{x}{2\sqrt{\alpha t}}\right)
		\]
		where $ T_s $ is the constant temperature during a cold period $ T_i $ is the initial soil temperature before the cold snap, $ \alpha $ is the thermal conductivity (in meters$ ^2 $) and 
		\[
			\erf(t) = \frac{2}{\sqrt{2}} \int_0^t e^{-s^2}\dd s
		\]
		Assume that $ T_i = 20 [\deg C]$, $ T_s = -15 [\deg C] $, $ \alpha = 0.138 \cdot 10^{-6} [$meters$ ^2 $ per second$ ] .$
		\item
		\item The sequence $ x_k $ produced by Newton's method is quadratically convergent to $ x_* $ with $ f(x_*) = 0 $, $ f'(x) \neq 0 $ and $ f''(x) $ continuous at $ x_* $.
		
		Let $ f(x) = (x-x_*)^p q(x) $ with $ p $ a positive integer with $ q $ twice continously differentiable and $ q(x_*) \neq 0 $. Note: $ f'(x_*) = 0. $ In the following sub-problems, let $ x_k, f_k = f(x_k), e_k = \abs{x_* - x_k}, $ etc.
		
		\begin{enumerate}[label = \roman*.]
			\item Prove that Newton's method converges linearly for $ f(x). $
			
			\item Consider the modified Newton iteration defined by 
			\[
				x_{k+1} = x_k - p \frac{f_k}{f_k'}.
			\]
			Prove that if $ x_k $ converges to $ x_* $, then the rate of convergence is quadratic.
				
			\item Write MATLAB codes for both Newton and modified Newton methods. Apply these to the function
			\[
				f(x) = (x-1)^5 e^x
			\]
			and compare the results. Use $ x_0 = 0 $ as a starting point.
		\end{enumerate}
	\end{enumerate}
\end{document}