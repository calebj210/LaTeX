\documentclass[a4paper,12pt]{article}

\usepackage[utf8]{inputenc}
\usepackage[left=0.5in,right=0.5in,top=1in,bottom=1in]{geometry}
\usepackage{amsmath,amssymb,amsfonts}
\usepackage{pgfplots,graphicx,calc,changepage}
\pgfplotsset{compat=newest}
\usepackage{enumitem}
\usepackage{fancyhdr}
\usepackage[colorlinks = true, linkcolor = blue]{hyperref}

% Syntax highlighting
\usepackage{listings}
\usepackage{xcolor}

\definecolor{codegreen}{rgb}{0.40,0.62,0.07}
\definecolor{codegray}{rgb}{0.5,0.5,0.5}
\definecolor{codeblue}{rgb}{0.09,0.57,0.73}
\definecolor{backcolour}{rgb}{1,1,1}

\lstdefinestyle{mystyle}{
    backgroundcolor=\color{backcolour},   
    commentstyle=\color{codegreen},
    keywordstyle=\color{magenta},
    numberstyle=\tiny\color{codegray},
    stringstyle=\color{codeblue},
    basicstyle=\ttfamily\small,
    breaklines=true,                     
    keepspaces=true,                 
    numbers=left,                    
    numbersep=5pt,                  
    showspaces=false,
    showstringspaces=false,
    showtabs=false,                  
    tabsize=4
}

\lstset{style=mystyle}

\newcommand{\nats}{\mathbb{N}}
\newcommand{\reals}{\mathbb{R}}
\newcommand{\rats}{\mathbb{Q}}
\newcommand{\ints}{\mathbb{Z}}
\newcommand{\comps}{\mathbb{C}}
\newcommand{\pols}{\mathcal{P}}
\newcommand{\cants}{\Delta\!\!\!\!\Delta}
\newcommand{\eps}{\varepsilon}
\newcommand{\st}{\backepsilon}
\newcommand{\abs}[1]{\left| #1 \right|}
\newcommand{\dom}[1]{\mathrm{dom}\left(#1\right)}
\newcommand{\for}{\text{ for }}
\newcommand{\dd}[1]{\mathrm{d}#1}
\newcommand{\spn}{\mathrm{sp}}
\newcommand{\nul}{\mathcal{N}}
\newcommand{\col}{\mathrm{col}}
\newcommand{\rank}{\mathrm{rank}}
\newcommand{\norm}[1]{\lVert #1 \rVert}
\newcommand{\inner}[1]{\left\langle #1 \right\rangle}
\newcommand{\pmat}[1]{\begin{pmatrix} #1 \end{pmatrix}}
\renewcommand{\and}{\text{ and }}

\newsavebox{\qed}
\newenvironment{proof}[2][$\square$]
    {\setlength{\parskip}{0pt}\par\textit{Proof:} #2\setlength{\parskip}{0.25cm}
        \savebox{\qed}{#1}
        \begin{adjustwidth}{\widthof{Proof:}}{}
    }
    {
        \hfill\usebox{\qed}\end{adjustwidth}
    }

\pagestyle{fancy}
\fancyhead{}
\lhead{Caleb Jacobs}
\chead{APPM 5600: Numerical Analysis I}
\rhead{Homework \#4}
\cfoot{}
\setlength{\headheight}{35pt}
\setlength{\parskip}{0.25cm}
\setlength{\parindent}{0pt}

\begin{document}
\begin{enumerate}[label = \arabic*.]
	\item Prove the following for $ x \in \comps^n $:
		\begin{enumerate}[label = (\alph*)]
			\item $ \norm{x}_\infty \leq \norm{x}_1 \leq n \norm{x}_\infty $.
				\begin{proof}{}
					For the first inequality, we have
					\begin{align*}
						\norm{x}_\infty &= \max_{1 \leq i \leq n} \abs{x_i} \\
						&\leq \sum_{i=1}^n \abs{x_i} \\
						&= \norm{x}_1.
					\end{align*}
					 For the second half of our inequality chain, we have
					 \begin{align*}
					 	\norm{x}_1 &= \sum_{i=1}^{n} \abs{x_i} \\
					 	&\leq \sum_{i=1}^{n} \left(\max_{1 \leq j \leq n} \abs{x_j}\right) \\
					 	&= n \max_{1 \leq j \leq n} \abs{x_j} \\
					 	&= n \norm{x}_\infty
					 \end{align*}
				 
				 	Thus, $ \norm{x}_\infty \leq \norm{x}_1 \leq n\norm{x}_\infty $.
				\end{proof}
			
			\item $ \norm{x}_\infty \leq \norm{x}_2 \leq \sqrt{n}\norm{x}_\infty $.
				\begin{proof}{}
					For the first inequality, we have
					\begin{align*}
						\norm{x}_\infty &= \max_{1 \leq i \leq n} \abs{x_i} \\
						&= \sqrt{\left(\max_{1 \leq i \leq n} \abs{x_i}\right)^2} \\
						&\leq \sqrt{\sum_{i=1}^{n} \abs{x_i}^2} \\
						&= \norm{x}_2.
					\end{align*}
					For the second half of our inequality chain, we have
					\begin{align*}
						\norm{x}_2 &= \sqrt{\sum_{i=1}^{n} \abs{x_i}^2} \\
						&\leq \sqrt{\sum_{i=1}^{n} \left(\max_{1 \leq j \leq n}\abs{x_j}\right)^2} \\
						&= \sqrt{n \left(\max_{1 \leq j \leq n} \abs{x_j}\right)^2} \\
						&= \sqrt{n} \max_{1 \leq j \leq n} \abs{x_j} \\
						&= n \norm{x}_\infty.
					\end{align*}
						Thus, $ \norm{x}_\infty \leq \norm{x}_2 \leq \sqrt{n} \norm{x}_\infty $.
				\end{proof}
			
			\item $ \norm{x}_2 \leq \norm{x}_1 \leq \sqrt{n}\norm{x}_2 $.
				\begin{proof}{}
					
				\end{proof}
		\end{enumerate}
\end{enumerate}
\end{document}