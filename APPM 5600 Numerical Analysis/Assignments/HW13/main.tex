\documentclass[a4paper,12pt]{article}

\usepackage[utf8]{inputenc}
\usepackage[left=0.5in,right=0.5in,top=1in,bottom=1in]{geometry}
\usepackage{amsmath,amssymb,amsfonts}
\usepackage{pgfplots,graphicx,calc,changepage}
\pgfplotsset{compat=newest}
\usepackage{enumitem}
\usepackage{fancyhdr}
\usepackage[colorlinks = true, linkcolor = black]{hyperref}

% Syntax highlighting
\usepackage{listings}
\usepackage{xcolor}

\definecolor{codegreen}{rgb}{0.40,0.62,0.07}
\definecolor{codegray}{rgb}{0.5,0.5,0.5}
\definecolor{codeblue}{rgb}{0.09,0.57,0.73}
\definecolor{backcolour}{rgb}{1,1,1}

\lstdefinestyle{mystyle}{
    backgroundcolor=\color{backcolour},   
    commentstyle=\color{codegreen},
    keywordstyle=\color{magenta},
    numberstyle=\tiny\color{codegray},
    stringstyle=\color{codeblue},
    basicstyle=\ttfamily\small,
    breaklines=true,                     
    keepspaces=true,                 
    numbers=left,                    
    numbersep=5pt,                  
    showspaces=false,
    showstringspaces=false,
    showtabs=false,                  
    tabsize=4
}

\lstset{style=mystyle}

\newcommand{\nats}{\mathbb{N}}
\newcommand{\reals}{\mathbb{R}}
\newcommand{\rats}{\mathbb{Q}}
\newcommand{\ints}{\mathbb{Z}}
\newcommand{\comps}{\mathbb{C}}
\newcommand{\pols}{\mathcal{P}}
\newcommand{\cants}{\Delta\!\!\!\!\Delta}
\newcommand{\eps}{\varepsilon}
\newcommand{\st}{\backepsilon}
\newcommand{\abs}[1]{\left| #1 \right|}
\newcommand{\dom}[1]{\mathrm{dom}\left(#1\right)}
\newcommand{\for}{\text{ for }}
\newcommand{\dd}[1]{\mathrm{d}#1}
\newcommand{\spn}{\mathrm{sp}}
\newcommand{\nul}{\mathcal{N}}
\newcommand{\col}{\mathrm{col}}
\newcommand{\rank}{\mathrm{rank}}
\newcommand{\norm}[1]{\lVert #1 \rVert}
\newcommand{\inner}[1]{\left\langle #1 \right\rangle}
\newcommand{\pmat}[1]{\begin{pmatrix} #1 \end{pmatrix}}
\renewcommand{\and}{\text{ and }}

\newsavebox{\qed}
\newenvironment{proof}[2][$\square$]
    {\setlength{\parskip}{0pt}\par\textit{Proof:} #2\setlength{\parskip}{0.25cm}
        \savebox{\qed}{#1}
        \begin{adjustwidth}{\widthof{Proof:}}{}
    }
    {
        \hfill\usebox{\qed}\end{adjustwidth}
    }

\pagestyle{fancy}
\fancyhead{}
\lhead{Caleb Jacobs}
\chead{APPM 5600: Numerical Analysis I}
\rhead{Homework \#13}
\cfoot{}
\setlength{\headheight}{35pt}
\setlength{\parskip}{0.25cm}
\setlength{\parindent}{0pt}

\begin{document}
\begin{enumerate}[label = \arabic*.]
	\item 
		Derive a quadrature based on the cubic Hermite interpolating polynomial with data $ f(a) $, $ f(b) $, $ f'(a) $, and $ f'(b) $. Derive an upper bound on the error.
		
		Using the Hermite-Lagrange basis, we can construct our cubic Hermite polynomial as
		\[
			p(x) = f(a) H_a(x) + f(b) H_b(x) + f'(a) K_a(x) + f'(b) K_b(x)
		\]
		where
		\begin{align*}
			H_a(x) &= \left(1 - 2(x - a) \frac{1}{a - b}\right)\frac{(x - b)^2}{(a - b)^2} \\
			H_b(x) &= \left(1 - 2(x - b) \frac{1}{b - a}\right)\frac{(x - a)^2}{(b - a)^2} \\
			K_a(x) &= (x - a)\frac{(x - b)^2}{(a - b)^2} \\
			K_b(x) &= (x - b)\frac{(x - a)^2}{(b - a)^2}.
		\end{align*}
		Now, integrating $ p(x) $ over our interval, $ [a, b] $, we obtain our quadrature as
		\begin{align*}
			\int_a^b f(x) \;\dd x &\approx \int_a^b p(x) \;\dd x \\
			&= f(a) \int_a^b H_a(x) \;\dd x + f(b) \int_a^b H_b(x) \;\dd x + f'(a) \int_a^b K_a(x) \;\dd x + f'(b) \int_a^b K_b(x) \;\dd x \\
			&= f(a) \frac{b - a}{2} + f(b) \frac{b - a}{2} + f'(a) \frac{(a - b)^2}{12} - f'(b) \frac{(a - b)^2}{12} \\
			&= \boxed{(f(a) + f(b)) \frac{b - a}{2} + (f'(a) - f'(b)) \frac{(a - b)^2}{12}.}
		\end{align*}
		Now, assuming $ f \in C^4 [a,b] $, we can get an error bound for this quadrature by integrating the Hermite interpolant error as
		\begin{align*}
			E = \int_{a}^{b} \abs{f(x) - p(x)} \;\dd x &= \int_{a}^{b} \abs{\frac{f^{(4)}(\eta_x)}{4!} (x - a)^2 (x - b)^2} \;\dd x & \text{for some $ \eta_x \in [a, b] $} \\
			&\leq \frac{M}{24} \int_{a}^{b} (x - a)^2 (x - b)^2 \;\dd x & \text{where $ M = \max_{\eta \in [a,b]} \abs{f^{(4)}(\eta)} $} \\
			&= \frac{M}{24} (\frac{(b - a)^5}{30}) \\
			&= \boxed{\frac{M (b - a)^5}{720}.}
		\end{align*}
	
	\newpage
	\item 
		Assume the error in an integration formula has the asymptotic expansion
		\[
			I - I_n = \frac{C_1}{n \sqrt{n}} + \frac{C_2}{n^2} + \frac{C_3}{n^2\sqrt{n}} + \cdots.
		\]
		Generalize the Richardson extrapolation process to obtain an estimate of $ I $ with an error on the order $ \frac{1}{n^2 \sqrt{n}} $. Assume that three values $ I_n, I_{n / 2}, $ and $ I_{n / 4} $ have been computed.
		
		From the error formula, we have the three equations
		\begin{align}
			I &= I_n  + \frac{C_1}{n \sqrt{n}} + \frac{C_2}{n^2} + \frac{C_3}{n^2\sqrt{n}} + \cdots \label{equ:1} \\
			I &= I_{n / 2}  + 2\sqrt{2} \frac{C_1}{n \sqrt{n}} + 4\frac{C_2}{n^2} + 4\sqrt{2}\frac{C_3}{n^2\sqrt{n}} + \cdots \label{equ:2} \\
			I &= I_{n / 4}  + 8\frac{C_1}{n \sqrt{n}} + 16\frac{C_2}{n^2} + 32\frac{C_3}{n^2\sqrt{n}} + \cdots. \label{equ:3}
		\end{align}
		Using these three equations, we want to eliminate the $ C_1 $ and $ C_2 $ error terms which we can do by reducing
		\[
			\pmat{1 & 2\sqrt{2} & 8 \\ 1 & 4 & 16} \sim \pmat{1 & 0 & -8\sqrt{2} \\ 0 & 1 & 2(\sqrt{2} + 2)}
		\]
		which tells us that 
		\[
			8\sqrt{2}\eqref{equ:1} - 2(\sqrt{2} + 2)\eqref{equ:2} + \eqref{equ:3}
		\] 
		will eliminate our desired error terms. So, we have the equation
		\[
			(8\sqrt{2} - 2(\sqrt{2} + 2) + 1) I = 8\sqrt{2}I_n - 2(\sqrt{2} + 2)I_{n/2} + I_{n/4} + (16 - 8\sqrt{2})\frac{C_3}{n^2 \sqrt{n}}
		\]
		which implies
		\[
			I = \frac{8\sqrt{2}I_n - 2(\sqrt{2} + 2)I_{n/2} + I_{n/4}}{8\sqrt{2} - 2(\sqrt{2} + 2) + 1} + O\left(\frac{1}{n^2\sqrt{n}}\right).
		\]
		So if we use the integration formula $ I' $ defined as
		\[
			\boxed{I' = \frac{8\sqrt{2}I_n - 2(\sqrt{2} + 2)I_{n/2} + I_{n/4}}{8\sqrt{2} - 2(\sqrt{2} + 2) + 1}}
		\]
		we get our desired error
		\[
			\boxed{I - I' = O\left(\frac{1}{n^2\sqrt{n}}\right)}.
		\]
		
		\newpage
		\item
			Let $ n \geq 0 $.
			\begin{enumerate}[label = (\roman*)]
				\item Give a formula for the Gauss quadrature points $ x_j, j = 0, \ldots, n $, needed for the weight function $ w(x) = \frac{1}{\sqrt{1 - x^2}} $ on the interval $ [-1, 1] $.
				
				First, note that the Chebysheb polynomials are orthogonal under the given weight function. So, to find our nodes, $ x_j $, we need to find the roots of the $ n + 1 $ Chebyshev polynomial which is given by 
				\[
					T_{n + 1} (x) = \cos((n + 1)\arccos(x)).
				\]
				To find the roots of $ T_{n + 1} $, we need
				\[
					(n + 1) \arccos(x) = \frac{\pi}{2} + j \pi
				\]
				for any integer $ j $. So, we must have
				\[
					\boxed{x_j = \cos\left(\frac{\left(j + \frac{1}{2}\right)\pi}{n + 1}\right).}
				\]
				So we don't have overlapping $ x_j $, restrict $ j $ to $ j = 0, \ldots, n $.
				
				\item Show that for positive integers $ n $, 
				\[
					\sum_{j = 0}^{n} \cos((2j + 1) \theta) = \frac{\sin((2n + 2) \theta)}{2 \sin(\theta)},
				\]
				unless $ \theta $ is a multiple of $ \pi $. What is the value of the sum when $ \theta $ is a multiple of $ \pi $?
				
				To begin showing our sum, note that from a product-to-sum identity, we have
				\[
					2 \sin(\theta)\cos((2j + 1) \theta) = \sin((2j + 2) \theta) - \sin(2j \theta).
				\]
				Using this identity, we have the telescoping sum
				\begin{align*}
					\sum_{j = 0}^{n} 2 \sin(\theta) \cos((2j + 1)\theta)  =& +\sin(2 \theta) - 0 \\[-0.5cm]
					& + \sin(4\theta) - \sin(2\theta) \\
					& + \sin(6\theta) - \sin(4\theta) \\
					& + \cdots \\
					& + \sin(2n\theta) - \sin((2n - 2)\theta) \\
					& + \sin((2n + 2)\theta) - \sin(2n\theta) \\
					=& \sin((2n + 2)\theta)
				\end{align*}
				which implies
				\[
					\sum_{j = 0}^{n} 2 \sin(\theta) \cos((2j + 1)\theta) = \sin((2n + 2)\theta)
				\]
				or solving for our desired sum,
				\[
					\boxed{\sum_{j = 0}^{n} \cos((2j + 1)\theta) = \frac{\sin((2n + 2)\theta)}{2 \sin(\theta)}}
				\]
				If $ \theta $ is a multiple of $ \pi $, then
				\[
					\boxed{\sum_{j = 0}^{n} \cos((2j + 1)\theta) = 
						\begin{cases}
							 n + 1, & \theta \text{ is an even multiple} \\ 
							-n - 1, & \theta \text{ is an odd multiple}
						\end{cases}.}
				\]
				
				\item Suppose
				\[
					T_n(x) = \cos(n \arccos(x)), \quad x \in [-1, 1].
				\]
				Then, for integers $ k = 1, \ldots, n $, we have
				\begin{align*}
					\sum_{j = 0}^{n} T_k(x_j) &= \sum_{j = 0}^{n} \cos\left(k \arccos\left(\cos\left(\frac{\left(j + \frac{1}{2}\right)\pi}{n + 1}\right)\right)\right) \\
					&= \sum_{j = 0}^{n} \cos\left(k \left(\frac{\left(j + \frac{1}{2}\right)\pi}{n + 1}\right)\right) \\
					&= \sum_{j = 0}^{n} \cos\left((2j + 1)\frac{k \pi}{2n + 2}\right) \\
					&= \frac{\sin\left((2n + 2)\frac{k \pi}{2n + 2}\right)}{2 \sin\left(\frac{k \pi}{2n + 2}\right)} \\
					&=\frac{\sin(k \pi)}{2 \sin\left(\frac{k \pi}{2n + 2}\right)} \\
					&= 0.
				\end{align*}
				However,
				\begin{align*}
					\int_{-1}^{1} \frac{T_k(x)}{\sqrt{1 - x^2}} \;\dd x &= \int_{-1}^{1} \frac{\cos(k \arccos(x))}{\sqrt{1 - x^2}} \;\dd x \\
					&= \frac{\sin{k \pi}}{k} \\
					&= 0.
				\end{align*}
				So
				\[
					\boxed{\sum_{j = 0}^{n} T_k(x_j) = \int_{-1}^{1} \frac{T_k(x)}{\sqrt{1 - x^2}} \;\dd x.}
				\]
				
				Next, we have
				\[
					\sum_{j = 0}^{n} T_0(x_j) = \sum_{j = 0}^{n} \cos(0) = \sum_{j = 0}^{n} 1 = n + 1.
				\]
				Similarly
				\[
					\frac{n + 1}{\pi} \int_{-1}^{1} \frac{T_0(x)}{\sqrt{1 - x^2}} \;\dd x = \frac{n + 1}{\pi} \int_{-1}^{1} \frac{1}{\sqrt{1 - x^2}} \;\dd x = \frac{n + 1}{\pi} \pi = n + 1.
				\]
				So
				\[
					\boxed{\sum_{j = 0}^{n} T_0(x_j) = \frac{n + 1}{\pi} \int_{-1}^{1} \frac{T_0(x)}{\sqrt{1 - x^2}} \;\dd x.}
				\]
				
				\item Now, lets compute the quadrature weights with the weight function
				\[
					w(x) = \frac{1}{\sqrt{1 - x^2}}
				\]
				on the interval $ (-1, 1) $.
				
				\begin{align*}
					W_k = \int_{-1}^{1} \frac{\phi(x)}{\sqrt{1 - x^2}} \;\dd x &= \int_{-1}^{1} \frac{\sum_{s = 0}^{n} C_s T_s (x)}{\sqrt{1 - x^2}} \\
					&= \frac{\pi}{n + 1} C_0 \sum_{j = 1}^{n} T_0 (x_j) + \sum_{s = 1}^{n} \sum_{j = 0}^{n} C_sT_s(x_j) \\
					&= \frac{\pi}{n + 1}.
				\end{align*}
			\end{enumerate}
		
		\newpage
		\item
			The gamma function is defined by the formula
			\[
				\Gamma(x) = \int_{0}^{\infty} t^{x - 1} e^{-t} \;\dd t, \quad x > 0.
			\]
			A simple program to approximate $ \Gamma(x) $ using trapezoidal rule applied to a truncated line.
			
			\rule{\textwidth}{.4pt}
				\lstinputlisting[language = matlab]{code/Problem_3.m}
			\rule{\textwidth}{.4pt}	
			
			\begin{enumerate}[label = (\alph*)]
				\item My idea for choosing the interval to integrate over involved figuring out when the integrand was \emph{small enough} compared to my desired tolerance. Because the integrand decays to zero as $ t -> 0 $ for all $ x $, we are able to find some cutoff so that the tail end of the integral does not affect the overall numerical result to our desired tolerance. 
				
				\item In playing with MATLAB, the quad routine in MATLAB used less function evaluations to obtain the same error.
				
				\item Using the given Gauss-Laguerre quadrature code to obtain our weights and nodes allowed the integral for $ \Gamma $ to converge much more rapidly and with less work!
			\end{enumerate}
		
		\newpage
		\item
			Gaussian quadrature
			\[
				\int_{-1}^{1} f(x) \;\dd x = \sum_{k = 0}^{n} w_k f(x_k)
			\]
			with nodes including the endpoints ($ x_0 = -1 $ and $ x_n = 1 $) using Legendre polynomials are called \emph{Gauss-Legendre-Lobatto} quadratures.
			\begin{enumerate}[label = (\roman*)]
				\item Show that if the interior nodes $ x_1, \ldots, x_{n - 1} $ in the quadrature are given by the roots of $ p'_n(x) $ where $ p_n(x) $ denotes the $ n $-th degree Legendre polynomial, then the quadrature is exact for polynomials up to degree $ 2n - 1 $.
				
				Before we begin, note that $ (x^2 - 1)p'_n(x) = \frac{x}{n}p_n(x) - \frac{1}{n}p_{n-1}(x) $. Then, for any $ h \in \pols_{2n - 1} $ and some $ q,r \in \pols_{n - 2} $, we have
				\[
					h(x) = \left(\frac{x}{n}p_n(x) - \frac{1}{n}p_{n-1}(x)\right) q(x) + r(x).
				\]
				Now, define $ \bar{q}(x) = x q(x) $; note that $ \bar{q} \in \pols_{n - 1} $. Then
				\begin{align*}
					\int_{-1}^{1} h(x) \;\dd x &= \int_{-1}^{1} \left(\frac{x}{n}p_n(x) - \frac{1}{n}p_{n-1}(x)\right) q(x) + r(x) \;\dd x \\
					&= \underbrace{\int_{-1}^{1} \frac{1}{n}p_n(x) \bar{q}(x) - \frac{1}{n}p_{n-1}(x) q(x) \;\dd x}_{\text{0 $ \because $ of orthogonality of $ p_n $ and $ p_{n - 1} $}} + \int_{-1}^{1} r(x) \;\dd x \\
					&= \underbrace{\int_{-1}^{1} (x^2 - 1)p'_n(x) q(x) \;\dd x}_{\text{still 0}} + \int_{-1}^{1} r(x) \;\dd x \\
				\end{align*}
				which implies that if we chose our boundary nodes $ x_0 = -1, x_n = 1 $ and every interior node to be a root of $ p'_n(x) $, by construction, our quadrature will be exact for polynomials $ h \in \pols_{2n - 1} $.
			
				\item Find the 4-point Gauss-Legendre-Lobatto quadrature for the integral $ \int_{-1}^{1} f(x) \;\dd x $.
				
				First, let's find the interior nodes. We know 
				\[
					p_3(x) = \frac{1}{2} (5x^3 - 3x) \implies p'_3(x) = \frac{1}{2} (15 x^2 - 3)
				\]	
				So, $ p'_3(x) = 0 $ when $ x = \pm \frac{1}{\sqrt{5}} $. Thus, our nodes are $ x_0 = -1, x_1 = -\frac{1}{\sqrt{5}}, x_2 = \frac{1}{\sqrt{5}}, $ and $ x_3 = 1 $. Next, we can compute our weights as
				\begin{align*}
					w_0 &= \int_{-1}^{1} \frac{\left(x + \frac{1}{\sqrt{5}}\right) \left(x - \frac{1}{\sqrt{5}}\right) (x - 1)}{\left(-1 + \frac{1}{\sqrt{5}}\right) \left(-1 - \frac{1}{\sqrt{5}}\right) (-1 - 1)} = \frac{1}{6} \\
					w_1 &= \int_{-1}^{1} \frac{(x + 1) \left(x - \frac{1}{\sqrt{5}}\right) (x - 1)}{\left(-\frac{1}{\sqrt{5}} + 1\right) \left(-\frac{1}{\sqrt{5}} - \frac{1}{\sqrt{5}}\right) \left(-\frac{1}{\sqrt{5}} - 1\right)} = \frac{5}{6} \\
					w_2 &= \int_{-1}^{1} \frac{(x + 1) \left(x + \frac{1}{\sqrt{5}}\right) (x - 1)}{\left(\frac{1}{\sqrt{5}} + 1\right) \left(\frac{1}{\sqrt{5}} + \frac{1}{\sqrt{5}}\right) \left(\frac{1}{\sqrt{5}} - 1\right)} = \frac{5}{6} \\
					w_3 &= \int_{-1}^{1} \frac{(x + 1) \left(x + \frac{1}{\sqrt{5}}\right) \left(x - \frac{1}{\sqrt{5}}\right)}{(1 + 1) \left(1 + \frac{1}{\sqrt{5}}\right) \left(1 - \frac{1}{\sqrt{5}}\right)} = \frac{1}{6}.
				\end{align*}
				Therefore, our quadrature is
				\[
					\boxed{\int_{-1}^{1} f(x) \;\dd x \approx \frac{1}{6} f(-1) + \frac{5}{6} f\left(-\frac{1}{\sqrt{5}}\right) + \frac{5}{6} f\left(\frac{1}{\sqrt{5}}\right) + \frac{1}{6} f(1).}
				\]
			\end{enumerate} 
\end{enumerate}
\end{document}