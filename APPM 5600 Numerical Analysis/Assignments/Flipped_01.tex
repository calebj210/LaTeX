\documentclass[a4paper,12pt]{article}

\usepackage[utf8]{inputenc}
\usepackage[left=0.5in,right=0.5in,top=1in,bottom=1in]{geometry}
\usepackage{amsmath,amssymb,amsfonts}
\usepackage{algorithm, algpseudocode}
\usepackage{pgfplots,graphicx,calc,changepage}
\pgfplotsset{compat=newest}
\usepackage{enumitem}
\usepackage{fancyhdr}
\usepackage[colorlinks = true, linkcolor = blue]{hyperref}

\newcommand{\nats}{\mathbb{N}}
\newcommand{\reals}{\mathbb{R}}
\newcommand{\rats}{\mathbb{Q}}
\newcommand{\ints}{\mathbb{Z}}
\newcommand{\pols}{\mathcal{P}}
\newcommand{\cants}{\Delta\!\!\!\!\Delta}
\newcommand{\eps}{\varepsilon}
\newcommand{\st}{\backepsilon}
\newcommand{\abs}[1]{\left| #1 \right|}
\newcommand{\dom}[1]{\mathrm{dom}\left(#1\right)}
\newcommand{\for}{\text{ for }}
\newcommand{\dd}[1]{\mathrm{d}#1}
\newcommand{\spn}{\mathrm{sp}}
\newcommand{\nul}{\mathcal{N}}
\newcommand{\col}{\mathrm{col}}
\newcommand{\rank}{\mathrm{rank}}
\newcommand{\norm}[1]{\lVert #1 \rVert}
\newcommand{\inner}[1]{\left\langle #1 \right\rangle}
\newcommand{\pmat}[1]{\begin{pmatrix} #1 \end{pmatrix}}
\renewcommand{\and}{\text{ and }}
\DeclareMathOperator{\sign}{sign}

\newsavebox{\qed}
\newenvironment{proof}[2][$\square$]
    {\setlength{\parskip}{0pt}\par\textit{Proof:} #2\setlength{\parskip}{0.25cm}
        \savebox{\qed}{#1}
        \begin{adjustwidth}{\widthof{Proof:}}{}
    }
    {
        \hfill\usebox{\qed}\end{adjustwidth}
    }

\pagestyle{fancy}
\fancyhead{}
\lhead{Caleb Jacobs}
\chead{APPM 5600: Numerical Analysis I}
\rhead{Flipped Day: Bisection}
\cfoot{}
\setlength{\headheight}{35pt}
\setlength{\parskip}{0.25cm}
\setlength{\parindent}{0pt}

\begin{document}
Suppose we have a function $f(x)$ that is smooth for all $x \in \reals$.
\begin{itemize}
    \item One way that we can determine if a root is in an interval is to apply the the IVT. To do so, evaluate $f(x)$ at the end points of the interval and if the sign of $f$ differs at each end point, then by the IVT, there must be a root inside of the interval. 
    
    Now, there is a drawback to this approach. The most apparent draw back is that a function could have endpoints that are the same sign and still dip down to the x-axis to form a root in the interval.
    
    \item In algebraic terms, suppose we want to check if there is a root to $f$ in the interval $[a, b]$. If
    \[
        \sign(f(a)) \neq \sign(f(b))
    \]
    then there is a root to $f$ in the interval $[a, b]$. We can actually simplify the above algebraic check to 
    \[
        f(a) \cdot f(b) < 0.
    \]
\end{itemize}

\begin{algorithm}
    \caption{Bisection Algorithm}
    \begin{algorithmic}[1]
    \Procedure{bisect}{$a, b, \varepsilon, maxIts$}
        \State $i \gets 1$
        \While{$i \leq maxIts$}
            \State $c \gets (b + a) / 2$ \Comment{Midpoint of interval}
            \If{$f(c) = 0$ or $(b - a) / 2 < \varepsilon$}
                \State \textbf{return} $c$ \Comment{Root found so return it}
            \EndIf
            
            \State $i \gets i + 1$
            \If{$\sign(f(c)) = \sign(f(a))$}
                \State $a \gets c$ \Comment{Take right interval}
            \Else
                \State $b \gets c$ \Comment{Take left interval}
            \EndIf
        \EndWhile
    \EndProcedure
    \end{algorithmic}
\end{algorithm}

\begin{itemize}
    \item If the bisection algorithm finds a root, we can compute the maximum number of steps by first realizing that the absolute error in our root guess gets halved at each iteration:
    \[
        \abs{x_0 - x} \leq \frac{1}{2^n}(b - a)
    \]
    where $x_0$ is the true root, $x$ is our current root guess, and $n$ is the number of iterations used thus far. Then, the maximum number of iterations $n$ to resolve the root to an accuracy of $\varepsilon$ satisfies
    \[
        \frac{1}{2^n}(b - a) \leq \varepsilon
    \]
    which implies
    \[
        n \geq \log_2\left(\frac{b - a}{\varepsilon}\right).
    \]
    Thus, the maximum number of iterations for convergence to the root within a tolerance of $\varepsilon$ is
    \[
        n = \left\lceil \log_2\left(\frac{b - a}{\varepsilon}\right) \right\rceil.
    \]
    
    \item After running the code on a few test problems, the behavior that I am seeing is expected and the found roots are withing my desired tolerance of $10^{-5}$. The only root that was not found was the very last root which makes sense because the interval did not contain a root of $\sin(x)$. Even more interesting is the fact that each terminated in the maximum possible steps using the calculation above!
\end{itemize}
\end{document}