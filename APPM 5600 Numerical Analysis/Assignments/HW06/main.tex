\documentclass[a4paper,12pt]{article}

\usepackage[utf8]{inputenc}
\usepackage[left=0.5in,right=0.5in,top=1in,bottom=1in]{geometry}
\usepackage{amsmath,amssymb,amsfonts}
\usepackage{pgfplots,graphicx,calc,changepage}
\pgfplotsset{compat=newest}
\usepackage{enumitem}
\usepackage{fancyhdr}
\usepackage[colorlinks = true, linkcolor = black]{hyperref}

% Syntax highlighting
\usepackage{listings}
\usepackage{xcolor}

\definecolor{codegreen}{rgb}{0.40,0.62,0.07}
\definecolor{codegray}{rgb}{0.5,0.5,0.5}
\definecolor{codeblue}{rgb}{0.09,0.57,0.73}
\definecolor{backcolour}{rgb}{1,1,1}

\lstdefinestyle{mystyle}{
    backgroundcolor=\color{backcolour},   
    commentstyle=\color{codegreen},
    keywordstyle=\color{magenta},
    numberstyle=\tiny\color{codegray},
    stringstyle=\color{codeblue},
    basicstyle=\ttfamily\small,
    breaklines=true,                     
    keepspaces=true,                 
    numbers=left,                    
    numbersep=5pt,                  
    showspaces=false,
    showstringspaces=false,
    showtabs=false,                  
    tabsize=4
}

\lstset{style=mystyle}

\newcommand{\nats}{\mathbb{N}}
\newcommand{\reals}{\mathbb{R}}
\newcommand{\rats}{\mathbb{Q}}
\newcommand{\ints}{\mathbb{Z}}
\newcommand{\comps}{\mathbb{C}}
\newcommand{\pols}{\mathcal{P}}
\newcommand{\cants}{\Delta\!\!\!\!\Delta}
\newcommand{\eps}{\varepsilon}
\newcommand{\st}{\backepsilon}
\newcommand{\abs}[1]{\left| #1 \right|}
\newcommand{\dom}[1]{\mathrm{dom}\left(#1\right)}
\newcommand{\for}{\text{ for }}
\newcommand{\dd}[1]{\mathrm{d}#1}
\newcommand{\spn}{\mathrm{sp}}
\newcommand{\nul}{\mathcal{N}}
\newcommand{\col}{\mathrm{col}}
\newcommand{\rank}{\mathrm{rank}}
\newcommand{\norm}[1]{\lVert #1 \rVert}
\newcommand{\inner}[1]{\left\langle #1 \right\rangle}
\newcommand{\pmat}[1]{\begin{pmatrix} #1 \end{pmatrix}}
\renewcommand{\and}{\text{ and }}

\newsavebox{\qed}
\newenvironment{proof}[2][$\square$]
    {\setlength{\parskip}{0pt}\par\textit{Proof:} #2\setlength{\parskip}{0.25cm}
        \savebox{\qed}{#1}
        \begin{adjustwidth}{\widthof{Proof:}}{}
    }
    {
        \hfill\usebox{\qed}\end{adjustwidth}
    }

\pagestyle{fancy}
\fancyhead{}
\lhead{Caleb Jacobs}
\chead{APPM 5600: Numerical Analysis I}
\rhead{Homework \#6}
\cfoot{}
\setlength{\headheight}{35pt}
\setlength{\parskip}{0.25cm}
\setlength{\parindent}{0pt}

\begin{document}
\begin{enumerate}[label = \arabic*)]
	\item In class, we showed that
	\begin{equation}
		p_{k+1} = r_{k+1}- \frac{\inner{p_k,r_{k+1}}_A}{\norm{p_k}_A^2}p_k. \label{equ:1}
	\end{equation}
	\begin{enumerate}[label = (\alph*)]
		\item Using the fact that $ r_{k+1} = r_k - \alpha_k A p_k $ and $ r_{k+1}^T r_k = 0 $, show that $ \inner{p_k, r_{k+1}}_A = -\frac{\norm{r_{k+1}}_2^2}{\alpha_k} $.
		
		\begin{align*}
			0 = r_{k+1}^T r_k &= r_{k+1}^T (r_{k+1} + \alpha_k A p_k) \\
			&= r_{k+1}^T r_{k+1} + \alpha_k r_{k+1} A p_k \\
			\implies r_{k+1} A p_k &= -\frac{r_{k+1}^T r_{k+1}}{\alpha_k}
		\end{align*}
		which implies
		\[
			\inner{p_k, r_{k+1}}_A = -\frac{\norm{r_{k+1}}_2^2}{\alpha_k}.
		\]
		
		\item Rewrite $ \norm{p_k}_A^2 $ in terms of $ r_k $ and $ \alpha_k $.
		
		\begin{align*}
			\norm{p_k}_A^2 &= p_k^T A p_k \\
			&= \left(r_k - \frac{\inner{p_{k-1}, r_k}}{\norm{p_{k-1}}_A^2} p_{k-1} \right)^T \frac{1}{\alpha_k}(r_k - r_{k+1}) \\
			&= \frac{1}{\alpha_k}(r_k^T r_k - r_k^Tr_{k+1}) \quad \text{because $ p_{k-1} $ is orthogonal to $ r_k $ and $ r_{k+1} $} \\
			&= \frac{1}{\alpha_k}r_k^T r_k \\
			&= \frac{1}{\alpha_k} \norm{r_k}_2^2.
		\end{align*}
		
		\item Plug these expressions into \eqref{equ:1} to get a technique for evaluating the next basis vector for the residual space without any applications of the matrix $ A $.
		\begin{align*}
			p_{k+1} &= r_{k+1} - \left(-\frac{\norm{r_{k+1}}_2^2}{\alpha_k}\right)\left(\frac{\alpha_k}{\norm{r_k}_2^2}\right) p_k \\
			&= r_{k+1} + \left(\frac{\norm{r_{k+1}}_2}{\norm{r_k}_2}\right)^2 p_k.
		\end{align*}
	\end{enumerate}

	\item Consider a sparse $ 500 \times 500 $ matrix $ A $ constructed as follows.
	\begin{itemize}[topsep = 0pt]
		\item Put a $ 1 $ in each diagonal entry.
		\item In each off-diagonal entry put a random number from the uniform distribution on $ [-1,1] $ but make sure to maintain symmetry. Then replace each off-diagonal entry with $ \abs{A_{ij}} > \tau $ by $ 0 $, where $ \tau $ for $ \tau = 0.01, 0.05, 0.1, $ and $ 0.2 $.
	
	\end{itemize}
		
	Take the right hand side to be a random vector $ b $ and set the tolerance to $ 10^{-10} $.
	\begin{enumerate}[label = (\alph*)]
		\item Write the Steepest Descent (SD) and Conjugate Gradient (CG) solver.
		
		\item Apply SD to solve each of the linear systems ad plot the residual for each iteration $ \norm{r_n} $ versus the iteration $ n $ on a \emph{semilogy} scale. 
		
		\item Apply CG to solve each of the linear systems ad plot the residual for each iteration $ \norm{r_n} $ versus the iteration $ n $ on a \emph{semilogy} scale. 
		
		\item What do you observe about the convergence of these methods? If the methods do not converge for any choices of $ \tau $ explain what's happening.
		
		\item How do the residual compare with the error bounds provided in class?
	\end{enumerate}

	\item Suppose CG is applied to a symmetric positive definite matrix $ A $ with the result $ \norm{e_0}_A = 1 $, and $ \norm{e_{10}}_A = 2 \cdot 2^{-10} $, where $ \norm{e_k}_A = \norm{x_k - x^*}_A $ and $ x^* $ is the true solution. Based solely on this data,
	\begin{enumerate}[label = (\alph*)]
		\item What bound can you give on $ \kappa(A) $?
		\item What bound can you give on $ \norm{e_{20}}_A $?
	\end{enumerate}

	\item Consider the task of solving the following system of nonlinear equations.
	\[
		f_1(x,y) = 3x^2 + 4y^2 - 1 = 0 \text{ and } f_2(x,y) = y^3 - 8x^3 - 1 = 0
	\]
	for the solution $ \alpha $ near $ (x,y) = (-0.5, 0.25) $.
	\begin{enumerate}[label = (\alph*)]
		\item Apply the fixed point iteration with
		\[
			g(x) = x - \pmat{0.016 & -0.17 \\ 0.52 & -0.26} \pmat{3x^2 + 4y^2 - 1 \\ y^3 - 8x^3 - 1}.
		\]
		You can use $ (-0.5, 0.25) $ as the initial condition. How many steps are needed to get an approximation to 7 digits of accuracy?
		
		\item Why is this a good choice for $ g(x) $.
	\end{enumerate}
\end{enumerate}
\end{document}