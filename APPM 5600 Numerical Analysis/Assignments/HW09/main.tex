\documentclass[a4paper,12pt]{article}

\usepackage[utf8]{inputenc}
\usepackage[left=0.5in,right=0.5in,top=1in,bottom=1in]{geometry}
\usepackage{amsmath,amssymb,amsfonts}
\usepackage{pgfplots,graphicx,calc,changepage}
\pgfplotsset{compat=newest}
\usepackage{enumitem}
\usepackage{fancyhdr}
\usepackage[colorlinks = true, linkcolor = blue]{hyperref}

% Syntax highlighting
\usepackage{listings}
\usepackage{xcolor}

\definecolor{codegreen}{rgb}{0.40,0.62,0.07}
\definecolor{codegray}{rgb}{0.5,0.5,0.5}
\definecolor{codeblue}{rgb}{0.09,0.57,0.73}
\definecolor{backcolour}{rgb}{1,1,1}

\lstdefinestyle{mystyle}{
    backgroundcolor=\color{backcolour},   
    commentstyle=\color{codegreen},
    keywordstyle=\color{magenta},
    numberstyle=\tiny\color{codegray},
    stringstyle=\color{codeblue},
    basicstyle=\ttfamily\small,
    breaklines=true,                     
    keepspaces=true,                 
    numbers=left,                    
    numbersep=5pt,                  
    showspaces=false,
    showstringspaces=false,
    showtabs=false,                  
    tabsize=4
}

\lstset{style=mystyle}

\newcommand{\nats}{\mathbb{N}}
\newcommand{\reals}{\mathbb{R}}
\newcommand{\rats}{\mathbb{Q}}
\newcommand{\ints}{\mathbb{Z}}
\newcommand{\comps}{\mathbb{C}}
\newcommand{\pols}{\mathcal{P}}
\newcommand{\cants}{\Delta\!\!\!\!\Delta}
\newcommand{\eps}{\varepsilon}
\newcommand{\st}{\backepsilon}
\newcommand{\abs}[1]{\left| #1 \right|}
\newcommand{\dom}[1]{\mathrm{dom}\left(#1\right)}
\newcommand{\for}{\text{ for }}
\newcommand{\dd}[1]{\mathrm{d}#1}
\newcommand{\spn}{\mathrm{sp}}
\newcommand{\nul}{\mathcal{N}}
\newcommand{\col}{\mathrm{col}}
\newcommand{\rank}{\mathrm{rank}}
\newcommand{\norm}[1]{\lVert #1 \rVert}
\newcommand{\inner}[1]{\left\langle #1 \right\rangle}
\newcommand{\pmat}[1]{\begin{pmatrix} #1 \end{pmatrix}}
\renewcommand{\and}{\text{ and }}

\newsavebox{\qed}
\newenvironment{proof}[2][$\square$]
    {\setlength{\parskip}{0pt}\par\textit{Proof:} #2\setlength{\parskip}{0.25cm}
        \savebox{\qed}{#1}
        \begin{adjustwidth}{\widthof{Proof:}}{}
    }
    {
        \hfill\usebox{\qed}\end{adjustwidth}
    }

\pagestyle{fancy}
\fancyhead{}
\lhead{Caleb Jacobs}
\chead{APPM 5600: Numerical Analysis I}
\rhead{Homework \#9}
\cfoot{}
\setlength{\headheight}{35pt}
\setlength{\parskip}{0.25cm}
\setlength{\parindent}{0pt}

\begin{document}
\begin{enumerate}[label = (\arabic*)]
	\item A circulant matrix of size $ (2n + 1) \times (2n + 1) $ has the form
	\[
		C = 
		\pmat{
			a_0 & a_1 & \cdots & \cdots & a_{2n} \\
			a_{2}  & a_0 & a_1 & \cdots & a_{2n - 1} \\
			a_{2n - 1} & a_{2n} & a_0 & \cdots & a_{2n - 2} \\
			\vdots & \vdots & \vdots & \ddots & \vdots \\
			a_1 & a_2 & \cdots & a_{2n} & a_0
		}.
	\]
	Furthermore, let $ S $ denote the matrix that shifts the index of a vector by 1. In this case, $ S $ will be a $ (2n + 1) \times (2n + 1) $ matrix of the form
	\[
		S = 
		\pmat{
			0 & 1 & 0 & 0 & \cdots & 0 \\
			0 & 0 & 1 & 0 & \cdots & 0 \\
			0 & 0 & 0 & 1 & \cdots & 0 \\
			\vdots & \vdots & \vdots & \vdots & \ddots & \vdots \\
			0 & 0 & \vdots & \cdots & \cdots & 1 \\
			1 & 0 & 0 & \cdots & \cdots &  0
		}.
	\]
	\begin{enumerate}[label = (\alph*)]
		\item Show that any circulant matrix can be written as a polynomial of the $ S $ matrix.
		
		To create a polynomial that makes any circulant matrix, notice that each $ a_0 $ in $ C $ is along the diagonal which can be formed by $ a_0 I = a_0 S^0 $. Next, notice that each $ a_1 $ in $ C $ can be given by $ a_1 S^1 $. Continuing this trend, we can see that each $ a_{2n} $ in $ C $ can be given by $ a_{2n} S^{2n} $. Finally, we can simply add up each term to get
		\[
			C = a_0 + a_1 S + a_2 S^2 + \cdots + a_{2n} S^{2n}
		\]
		which is a polynomial of $ S $.
		
		\item Let $ v^k $ denote the $ k $th orthogonal Fourier basis vector where the $ j $th entry of $ v^k $ is given by
		\[
			v_j^k = e^{\frac{2\pi i j k}{2n + 1}}.
		\]
		Prove that the vectors $ v^k $ are all the eigenvectors of the circulant matrix. Furthermore, what are the eigenvalues?
		
		To prove that the vectors $ v^k $ are the eigenvectors of the circulant matrix, let's check that each of these ``eigenvectors'' is actually an eigenvector of $ C $:
		\begin{align*}
			C v^k &= (a_0 + a_1 S + a_2 S^2 + \cdots + a_{2n} S^{2n}) v^k \\
			&= a_0v^k + a_1 Sv^k + a_2 S^2v^k + \cdots + a_{2n} S^{2n}v^k \\
			&= 
			\pmat{
				a_0v_0^k + a_1v_1^k + a_2 v_2^k + \cdots + a_{2n} v_{2n}^k \\
				a_0v_1^k + a_1v_2^k + a_2 v_3^k + \cdots + a_{2n} v_{0}^k \\
				\vdots\qquad\vdots\qquad\vdots\qquad\vdots\qquad\vdots \\
				a_0v_{2n}^k + a_1v_0^k + a_2 v_1^k + \cdots + a_{2n} v_{2n-1}^k
			} \\
			&= 
			\pmat{
				a_0e^{\frac{2\pi i 0 k}{2n + 1}} + a_1e^{\frac{2\pi i 1 k}{2n + 1}} + a_2 e^{\frac{2\pi i 2 k}{2n + 1}} + \cdots + a_{2n} e^{\frac{2\pi i 2n k}{2n + 1}} \\
				a_0e^{\frac{2\pi i 1 k}{2n + 1}} + a_1e^{\frac{2\pi i 2 k}{2n + 1}} + a_2 e^{\frac{2\pi i 3 k}{2n + 1}} + \cdots + a_{2n} e^{\frac{2\pi i 0 k}{2n + 1}} \\
				\vdots\qquad\vdots\qquad\vdots\qquad\vdots\qquad\vdots \\
				a_0e^{\frac{2\pi i 2n k}{2n + 1}} + a_1e^{\frac{2\pi i 0 k}{2n + 1}} + a_2 e^{\frac{2\pi i 1 k}{2n + 1}} + \cdots + a_{2n} e^{\frac{2\pi i (2n-1) k}{2n + 1}}
			} \\
			&= 
			\left(a_0 + a_1e^{\frac{2\pi i 1 k}{2n + 1}} + a_2 e^{\frac{2\pi i 2 k}{2n + 1}} + \cdots + a_{2n} e^{\frac{2\pi i 2n k}{2n + 1}}\right)
			\pmat{
				e^{\frac{2\pi i 0 k}{2n + 1}} \\
				e^{\frac{2\pi i 1 k}{2n + 1}} \\
				\vdots \\
				e^{\frac{2\pi i 2n k}{2n + 1}}
			} \\
			&= \left(a_0 + a_1e^{\frac{2\pi i 1 k}{2n + 1}} + a_2 e^{\frac{2\pi i 2 k}{2n + 1}} + \cdots + a_{2n} e^{\frac{2\pi i 2n k}{2n + 1}}\right) v^k.
		\end{align*}
		Thus, $ v^k $ is an eigenvector of $ C $ with eigenvalue
		\[
			\lambda_k = a_0 + a_1e^{\frac{2\pi i 1 k}{2n + 1}} + a_2 e^{\frac{2\pi i 2 k}{2n + 1}} + \cdots + a_{2n} e^{\frac{2\pi i 2n k}{2n + 1}}.
		\]
		Finally, because the collection of $ v^k $ forms a basis of dimension $ 2n + 1 $ and $ C $ is a $ (2n + 1) \times (2n + 1) $ matrix, the collection of $ v^k $ is all of the eigenvectors of $ C $.
	\end{enumerate}

	\newpage
	\item Let $ 0 \leq t_0 < t_1 < \cdots < t_{2n} < w\pi $ and consider the trigonometric polynomial interpolation problem: define
	\[
		l_j^{(n)}(t) = \prod_{\substack{k = 0 \\ k \neq j}}^{2n} \frac{\sin\left(\frac{1}{2} (t - t_k)\right)}{\sin\left(\frac{1}{2} (t_j - t_k)\right)}
	\]
	for $ j = 0, 1 \ldots, 2n $. It is easy to show that $ l_j(t_i) = \delta_{ij} $ for each $ j $.
	
	Show that $ l_j(t) $ is a trigonometric polynomial of degree less than or equal to $ n $. Then the solution of the trigonometric interpolation problem is given by
	\[
		p_n(t) = \sum_{j = 0}^{2n} f(t_j) l_j(t).
	\]
	
	To prove that the degree of our trig polynomial is less than or equal to $ n $ let's start with the base case when $ n = 1 $.
	\begin{align*}
		l_j^{(1)} &=  \prod_{\substack{k = 0 \\ k \neq j}}^{2} \frac{\sin\left(\frac{1}{2} (t - t_k)\right)}{\sin\left(\frac{1}{2} (t_j - t_k)\right)} \\ 
		&= \frac{\sin\left(\frac{1}{2} (t - t_l)\right)\sin\left(\frac{1}{2} (t - t_s)\right)}{\prod\limits_{\substack{k = 0 \\ k \neq j}}^{2} \sin\left(\frac{1}{2} (t_j - t_k)\right)} \\
		&= \frac{\cos\left(\frac{1}{2}(t_s - t_l)\right) - \cos\left(t - \frac{1}{2}(t_s + t_l)\right)}{2\prod\limits_{\substack{k = 0 \\ k \neq j}}^{2} \sin\left(\frac{1}{2} (t_j - t_k)\right)} \\
		&= \frac{\cos\left(\frac{1}{2}(t_s - t_l)\right) - \cos(t)\cos\left(\frac{1}{2}(t_s + t_l)\right) - \sin(t)\sin\left(\frac{1}{2}(t_s + t_l)\right)}{2\prod\limits_{\substack{k = 0 \\ k \neq j}}^{2} \sin\left(\frac{1}{2} (t_j - t_k)\right)} \\
		&= \sum_{k = 0}^{1} \left(a_k \cos(kt) + b_k \sin(kt)\right)
	\end{align*}
	for some constants $ a_n $ and $ b_n $. Therefore, $ l_j^{(1)} $ is a trig polynomial of degree 1 or less.
	
	Now, suppose $ l_j^{(i - 1)} $ is a trig polynomial of degree $ i - 1 $ or less. Then
	\begin{align*}
		l_j^{(i)} &= \prod_{\substack{k = 0 \\ k \neq j}}^{2i} \frac{\sin\left(\frac{1}{2} (t - t_k)\right)}{\sin\left(\frac{1}{2} (t_j - t_k)\right)} \\
	   	&= \left(\prod_{\substack{k = 0 \\ k \neq j}}^{2i - 2} \frac{\sin\left(\frac{1}{2} (t - t_k)\right)}{\sin\left(\frac{1}{2} (t_j - t_k)\right)}\right) \left(\prod_{\substack{k = 2i - 1 \\ k \neq j}}^{2i} \frac{\sin\left(\frac{1}{2} (t - t_k)\right)}{\sin\left(\frac{1}{2} (t_j - t_k)\right)}\right) \\
	   	&= l_j^{(i -1)}(t) \left(\prod_{\substack{k = 2i - 1 \\ k \neq j}}^{2i} \frac{\sin\left(\frac{1}{2} (t - t_k)\right)}{\sin\left(\frac{1}{2} (t_j - t_k)\right)}\right) \\
	   	&= l_j^{(i -1)}(t) \frac{\sin\left(\frac{1}{2} (t - t_l)\right) \sin\left(\frac{1}{2} (t - t_s)\right)}{\prod\limits_{\substack{k = 2i - 1 \\ k \neq j}}^{2i} \sin\left(\frac{1}{2} (t_j - t_k)\right)} \\
	   	&= l_j^{(i -1)}(t) \frac{\cos\left(\frac{1}{2}(t_s - t_l)\right) - \cos(t)\cos\left(\frac{1}{2}(t_s + t_l)\right) - \sin(t)\sin\left(\frac{1}{2}(t_s + t_l)\right)}{\prod\limits_{\substack{k = 2i - 1 \\ k \neq j}}^{2i} \sin\left(\frac{1}{2} (t_j - t_k)\right)}\\
	   	&= l_j^{(i -1)}(t) \left(\alpha + \beta \cos(t) + \gamma \sin(t)\right)
	\end{align*}
	for some constants $ \alpha, \beta $ and $ \gamma $. Then, 
	\begin{align*}
		l_j^{(i)} &= l_j^{(i -1)}(t) \left(\alpha + \beta \cos(t) + \gamma \sin(t)\right) \\
		&= \sum_{k = 0}^{i - 1} \left(a_k \cos(k t) + b_k \sin(k t) \left(\alpha + \beta \cos(t) + \gamma \sin(t)\right)\right) \\
		&= \sum_{k = 0}^{i - 1} A_k \cos((k - 1) t) + B_k \cos((k + 1)t) + C_k \sin((k - 1) t) + D_k \sin((k + 1) t) \\
		&= \sum_{k = 0}^{i} A'_k \cos(k t) + B'_k \sin(k t)
	\end{align*}
	for some constants $ A'_k $ and $ B'_k $. Therefore, $ l_j^{(i)} $ has degree less than or equal to $ i $. So, by induction $ l_j^n $ has degree less than or equal to $ n $.
\end{enumerate}
\end{document}