\documentclass[a4paper,12pt]{article}

\usepackage[utf8]{inputenc}
\usepackage[left=0.5in,right=0.5in,top=1in,bottom=1in]{geometry}
\usepackage{amsmath,amssymb,amsfonts}
\usepackage{pgfplots,graphicx,calc,changepage}
\pgfplotsset{compat=newest}
\usepackage{enumitem}
\usepackage{fancyhdr}
\usepackage[colorlinks = true, linkcolor = blue]{hyperref}

% Syntax highlighting
\usepackage{listings}
\usepackage{xcolor}

\definecolor{codegreen}{rgb}{0.40,0.62,0.07}
\definecolor{codegray}{rgb}{0.5,0.5,0.5}
\definecolor{codeblue}{rgb}{0.09,0.57,0.73}
\definecolor{backcolour}{rgb}{1,1,1}

\lstdefinestyle{mystyle}{
    backgroundcolor=\color{backcolour},   
    commentstyle=\color{codegreen},
    keywordstyle=\color{magenta},
    numberstyle=\tiny\color{codegray},
    stringstyle=\color{codeblue},
    basicstyle=\ttfamily\small,
    breaklines=true,                     
    keepspaces=true,                 
    numbers=left,                    
    numbersep=5pt,                  
    showspaces=false,
    showstringspaces=false,
    showtabs=false,                  
    tabsize=4
}

\lstset{style=mystyle}

\newcommand{\nats}{\mathbb{N}}
\newcommand{\reals}{\mathbb{R}}
\newcommand{\rats}{\mathbb{Q}}
\newcommand{\ints}{\mathbb{Z}}
\newcommand{\comps}{\mathbb{C}}
\newcommand{\pols}{\mathcal{P}}
\newcommand{\cants}{\Delta\!\!\!\!\Delta}
\newcommand{\eps}{\varepsilon}
\newcommand{\st}{\backepsilon}
\newcommand{\abs}[1]{\left| #1 \right|}
\newcommand{\dom}[1]{\mathrm{dom}\left(#1\right)}
\newcommand{\for}{\text{ for }}
\newcommand{\dd}[1]{\mathrm{d}#1}
\newcommand{\spn}{\mathrm{sp}}
\newcommand{\nul}{\mathcal{N}}
\newcommand{\col}{\mathrm{col}}
\newcommand{\rank}{\mathrm{rank}}
\newcommand{\norm}[1]{\lVert #1 \rVert}
\newcommand{\inner}[1]{\left\langle #1 \right\rangle}
\newcommand{\pmat}[1]{\begin{pmatrix} #1 \end{pmatrix}}
\renewcommand{\and}{\text{ and }}

\newsavebox{\qed}
\newenvironment{proof}[2][$\square$]
    {\setlength{\parskip}{0pt}\par\textit{Proof:} #2\setlength{\parskip}{0.25cm}
        \savebox{\qed}{#1}
        \begin{adjustwidth}{\widthof{Proof:}}{}
    }
    {
        \hfill\usebox{\qed}\end{adjustwidth}
    }

\pagestyle{fancy}
\fancyhead{}
\lhead{Caleb Jacobs}
\chead{APPM 5600: Numerical Analysis I}
\rhead{Homework \#5}
\cfoot{}
\setlength{\headheight}{35pt}
\setlength{\parskip}{0.25cm}
\setlength{\parindent}{0pt}

\begin{document}
\begin{enumerate}[label = \arabic*.]
	\item Let $ A \in \reals^{n \times n} $ be a tridiagonal matrix where the diagonal entries are given by $ a_j $ for $ j = 1, \ldots, n $, the lower diagonal entries are $ b_j $ for $ j = 2, \ldots, n $ and the upper diagonal entries are $ c_j $ for $ j = 1, \ldots, n-1 $.
	\begin{enumerate}[label = (\alph*)]
		\item For $ n = 3 $, derive the $ LU $ factorization of the matrix A.
		
		
	\end{enumerate}

	\item Consider the linear system
	\begin{align*}
		6x + 2y + 2z &= -2 \\
		2x + \frac{2}{3}y + \frac{1}{3} z &= 1 \\
		x + 2y - z &= 0
	\end{align*}

	\begin{enumerate}[label = (\alph*)]
		\item Verify that $ (x,y,z) = (2.6, -3.8, -5) $ is the exact solution.
			
			To verify the solution, let's first rewrite the LHS of the system and multiply by our vector to get
			\[
				\pmat{
					6 & 2 & 2 \\
					2 & \frac{2}{3} & \frac{1}{3} \\
					1 & 2 & -1
				}\pmat{
					2.6 \\
					-3.8 \\
					-5
				} = \pmat{
					-2 \\
					1 \\
					0
				}
			\]
			which shows the exact solution is given by $ (x,y,z) = (2.6, -3.8, -5) $.
			
			\item Let's create our augmented matrix and begin Gaussian elimination
			\begin{align*}
				\left(\begin{array}{c c c | c}
					6 & 2 & 2 & -2 \\
					2 & \frac{2}{3} & \frac{1}{3} & 1\\
					1 & 2 & -1 & 0
				\end{array}\right) 
				&\sim 
				\left(\begin{array}{c c c | c}
					1 & 0.3333 & 0.3333 & -0.3333 \\
					2 & 0.6667 & 0.3333 & 1 \\
					1 & 2 & -1 & 0
				\end{array}\right) \\
				&\sim
				\left(\begin{array}{c c c | c}
					1 & 0.3333 & 0.3333 & -2 \\
					0 & 0.0001 & -0.3333 & 1.666 \\
					0 & 0 & -1.333 & 0.3333
				\end{array}\right) \\ 
				&\sim
				\left(\begin{array}{c c c | c}
					1 & 0.3333 & 0.3333 & -2 \\
					0 & 1 & -3333 & 16660 \\
					0 & 0 & -1.333 & 0.3333
				\end{array}\right) \\ 
				&\sim
				\left(\begin{array}{c c c | c}
					1 & 0 & 1111 & -5554 \\
					0 & 1 & -3333 & 16660 \\
					0 & 0 & -1.333 & 0.3333
				\end{array}\right) \\ 
				&\sim
				\left(\begin{array}{c c c | c}
					1 & 0 & 1111 & -5554 \\
					0 & 1 & -3333 & 16660 \\
					0 & 0 & 1 & -0.2500
				\end{array}\right) \\ 
				&\sim
				\left(\begin{array}{c c c | c}
					1 & 0 & 0 & -5276 \\
					0 & 1 & 0 & 15820 \\
					0 & 0 & 1 & -0.2500
				\end{array}\right) 
			\end{align*}
	\end{enumerate}
\end{enumerate}
\end{document}