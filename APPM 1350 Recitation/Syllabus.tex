\documentclass[10pt]{report}

% Package imports
\usepackage{amsmath,amssymb}
\usepackage[paperwidth=8.5in,paperheight=11in,margin=1in]{geometry}
\usepackage{enumitem}
\usepackage[colorlinks = true, urlcolor = blue]{hyperref}

% Basic settings
\pagenumbering{gobble}
\setlength{\parindent}{0pt}
\setlength{\parskip}{6pt}

%%%%%%%%%%%%%%%%%%%
\begin{document}
\begin{center}
    \textbf{APPM 1350: Calculus I for Engineers Recitation Syllabus}

    \textbf{Teaching Assistant:} Caleb Jacobs \\
    \textbf{Email:} Caleb.Jacobs@colorado.edu \\
    \textbf{Personal Office hours:} TBA
\end{center}

\textbf{Goals of Recitation:} The purpose of recitation is deliberate practice on key concepts and topics from lecture, mathematical discourse and collaboration with your peers, and retrieval practice with immediate feedback (through quizzes).

\textbf{Office Hours:} In addition to my office hours, the full list of Calc I office hours staffed by all TAs is as follows. You may attend the office hours of any TA or instructor:
\begin{itemize}[topsep = 0pt]
	\item TBA: Should be on canvas soon.
\end{itemize}

\textbf{Homework:} 13 written assignments total, to be scanned and submitted to Gradescope.

\begin{itemize}[topsep = 0pt]
	\item \textbf{Grading Policy} Each homework will be graded out of 20 points
	\begin{itemize}[topsep = 0pt]
		\item 3 problems will be graded at random for 5 points each. Answers without proper justification will not receive full credit. The remaining 5 points will be given for completion of all other problems.
		\item Written homework must adhere to the APPM Homework style guide. The full set of guidelines is described here: \url{https://www.colorado.edu/amath/appm-homework-style-guide}. Failure to adhere to these guidelines will result in a zero on the homework assignment, which can be recovered by meeting with an APPM style tutor. These guidelines include, but are not limited to
		\begin{itemize}[topsep = 0pt]
		\item Your name, recitation section, due date, and homework number on the top left corner of the first page
		\item Your Student ID on the top right corner of the first page
		\item Uploading a \underline{.pdf scan} that is legible. Uploading a photo (.png, .jpeg, etc.) will result in a zero for the submission.
		\item Labeling all problems on the page and in the Gradescope submission
		\item Writing your solutions in order from left to right and top to bottom or in two columns.
		\item Presenting solutions in a clear and logical manner, appropriately using mathematical symbols.
		\end{itemize}
		
	\end{itemize}
	\item \textbf{Late Policy} Homework is due scanned and uploaded to Gradescope by 11:59pm on Mondays. You are allowed a one week extension on up to 2 Written HW Assignments. Subsequent late assignments will NOT be accepted.
\end{itemize}

\textbf{Quizzes:} 10 Quizzes total (lowest two dropped)

\begin{itemize}[topsep = 0pt]
	\item  Quizzes will take place during the last 10 minutes of recitation.  
	\item  Each quiz will be graded out of 10 points
	\item  You must scan and upload your quiz in Gradescope with a .pdf extension.  
	\item  You are expected to study in advance for quizzes. See the course schedule for the specific topics covered in each quiz.
	\item There are \textbf{NO} make-up quizzes.
\end{itemize}

\textbf{Correction Policy:} It is your responsibility to review homework assignments and quizzes within one week after they have been returned in class and to verify that the grades have been posted correctly in Canvas. After one week, no grade changes will be made.

\textbf{Course Syllabus:} Please refer to the APPM 1350 course syllabus, which can be found on Canvas, for more information on course policies.

\textbf{Academic Honesty:} You are free to collaborate with other people when doing the homework problems.  However, the work you turn in must be your own, i.e. please don't copy someone else's assignment!  See \url{https://www.colorado.edu/sccr/honor-code} for more information on CU's honor code.

\end{document}
