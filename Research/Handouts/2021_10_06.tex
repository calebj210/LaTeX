\documentclass[a4paper,11pt]{article}

\usepackage[utf8]{inputenc}
\usepackage[left=0.5in,right=0.5in,top=1in,bottom=1in]{geometry}
\usepackage{amsmath,amssymb,amsfonts}
\usepackage{pgfplots,graphicx,calc,changepage}
\pgfplotsset{compat=newest}
\usepackage{enumitem}
\usepackage{fancyhdr}
\usepackage[colorlinks = true, linkcolor = blue]{hyperref}

\newcommand{\nats}{\mathbb{N}}
\newcommand{\reals}{\mathbb{R}}
\newcommand{\rats}{\mathbb{Q}}
\newcommand{\ints}{\mathbb{Z}}
\newcommand{\pols}{\mathcal{P}}
\newcommand{\cants}{\Delta\!\!\!\!\Delta}
\newcommand{\eps}{\varepsilon}
\newcommand{\st}{\backepsilon}
\newcommand{\abs}[1]{\left| #1 \right|}
\newcommand{\dom}[1]{\mathrm{dom}\left(#1\right)}
\newcommand{\for}{\text{ for }}
\newcommand{\dd}[1]{\mathrm{d}#1}
\newcommand{\spn}{\mathrm{sp}}
\newcommand{\nul}{\mathcal{N}}
\newcommand{\col}{\mathrm{col}}
\newcommand{\rank}{\mathrm{rank}}
\newcommand{\norm}[1]{\lVert #1 \rVert}
\newcommand{\inner}[1]{\left\langle #1 \right\rangle}
\newcommand{\pmat}[1]{\begin{pmatrix} #1 \end{pmatrix}}
\renewcommand{\and}{\text{ and }}

\newsavebox{\qed}
\newenvironment{proof}[2][$\square$]
    {\setlength{\parskip}{0pt}\par\textit{Proof:} #2\setlength{\parskip}{0.25cm}
        \savebox{\qed}{#1}
        \begin{adjustwidth}{\widthof{Proof:}}{}
    }
    {
        \hfill\usebox{\qed}\end{adjustwidth}
    }

\pagestyle{fancy}
\fancyhead{}
\lhead{Caleb Jacobs}
\chead{Evolving Curves and Surfaces Handout}
\rhead{October 6, 2021}
\cfoot{}
\setlength{\headheight}{35pt}
\setlength{\parskip}{0.25cm}
\setlength{\parindent}{0pt}

\begin{document}
\section*{Introduction}
	Numerical methods for solving PDEs on evolving curves and surfaces has historically taken one of two paths. The first and most intuitive approach is to model your surface using a set of points on the surface. Then, to evolve in time, you simply push the points in space according to a governing PDE. The second method is to interpret your surface as a level-set to so some higher dimensional function and then evolve that function according to some level-set PDE. Both methods compliment each others weaknesses: where point based methods have high accuracy but resist difficult geometries, level-set methods handle difficult geometries but at the cost of accuracy and time. 
	
\section*{Research Path}
	The introduction of modern mesh-free RBF-FD based methods into evolving curves and surfaces has opened the door to many new methods. In the case of level-set methods, RBF-FD has made it easy to create high-order reinitialaztion schemes while having the freedom to have complex background node-sets that can adapt in time.
	
	However, using level-set methods limits the direct use of the powerful Ricci calculus and the Calculus of Moving Surfaces (CMS) formulations of evolving curves and surfaces. So, my current research with Dr. C\'ecile Piret and Jacob Blazejewski is focused on developing a method that can still harness the power of CMS while still utilizing the level-set method's ability to handle piecewise smooth manifolds and changing topologies. 
	
\subsection*{New Formulation}
	To harness the power of CMS, our new method locally initializes level-set like background nodes using constraints imposed by the tensor formulation of our surface. Then, using this background node set, we can march our ambient solution in time either using the level-set equation or another solver from CMS ideas. Once we have marched a bit in time, we can recover our evolved surface by using the Coul-Newton method to get a new node band our surface. Then, just like before, we can initialize a new background set with new constraints as needed.
	
	The benefit of this new method over level-set methods is that our initialization can be much more accurate and stable than reinitialaztion steps that are required by level-set methods.
	
\subsection*{Timeline}
	This project has been underway for some time and has had many paths ranging from RBF generated finite surface differences on static surfaces to RBF based  level-set methods. With our future research focusing on the interface between level-set methods and point based methods, 
\end{document}
