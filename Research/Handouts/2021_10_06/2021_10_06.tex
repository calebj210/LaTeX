\documentclass[a4paper,11pt]{article}

\usepackage[utf8]{inputenc}
\usepackage[left=0.5in,right=0.5in,top=1in,bottom=1in]{geometry}
\usepackage{amsmath,amssymb,amsfonts}
\usepackage{pgfplots,graphicx,calc,changepage}
\pgfplotsset{compat=newest}
\usepackage{enumitem}
\usepackage{fancyhdr}
\usepackage[colorlinks = true, linkcolor = blue]{hyperref}

\newcommand{\nats}{\mathbb{N}}
\newcommand{\reals}{\mathbb{R}}
\newcommand{\rats}{\mathbb{Q}}
\newcommand{\ints}{\mathbb{Z}}
\newcommand{\pols}{\mathcal{P}}
\newcommand{\pos}{\mathbf{R}}
\newcommand{\covt}{\dot{\nabla}}
\newcommand{\cants}{\Delta\!\!\!\!\Delta}
\newcommand{\eps}{\varepsilon}
\newcommand{\st}{\backepsilon}
\newcommand{\abs}[1]{\left| #1 \right|}
\newcommand{\dom}[1]{\mathrm{dom}\left(#1\right)}
\newcommand{\for}{\text{ for }}
\newcommand{\dd}{\mathrm{d}}
\newcommand{\pd}{\partial}
\newcommand{\spn}{\mathrm{sp}}
\newcommand{\nul}{\mathcal{N}}
\newcommand{\col}{\mathrm{col}}
\newcommand{\rank}{\mathrm{rank}}
\newcommand{\norm}[1]{\lVert #1 \rVert}
\newcommand{\inner}[1]{\left\langle #1 \right\rangle}
\newcommand{\pmat}[1]{\begin{pmatrix} #1 \end{pmatrix}}
\renewcommand{\and}{\text{ and }}

\newsavebox{\qed}
\newenvironment{proof}[2][$\square$]
    {\setlength{\parskip}{0pt}\par\textit{Proof:} #2\setlength{\parskip}{0.25cm}
        \savebox{\qed}{#1}
        \begin{adjustwidth}{\widthof{Proof:}}{}
    }
    {
        \hfill\usebox{\qed}\end{adjustwidth}
    }

\pagestyle{fancy}
\fancyhead{}
\lhead{Caleb Jacobs}
\chead{Evolving Curves and Surfaces Handout}
\rhead{October 6, 2021}
\cfoot{}
\setlength{\headheight}{35pt}
\setlength{\parskip}{0.25cm}
\setlength{\parindent}{0pt}

\begin{document}
\section*{Introduction}
	Numerical methods for solving PDEs on evolving curves and surfaces has historically taken one of two paths. The first and most intuitive approach is to model your surface using a set of points on the surface. Then, to evolve in time, you simply push the points in space according to a governing PDE. The second method is to interpret your surface as a level-set to so some higher dimensional function and then evolve that function according to some level-set PDE. Both methods compliment each others weaknesses: where point based methods have high accuracy but resist difficult geometries, level-set methods handle difficult geometries but at the cost of accuracy and time. 
	
\section*{Research Path}
	The introduction of modern mesh-free RBF-FD based methods into evolving curves and surfaces has opened the door to many new methods. In the case of level-set methods, RBF-FD has made it easy to create complex background node sets that can adapt in time.
	
	However, using level-set methods limits the direct use of the powerful Ricci calculus and the Calculus of Moving Surfaces (CMS) formulations of evolving curves and surfaces. So, my current research with Dr. C\'ecile Piret and Jacob Blazejewski is focused on developing a method that can still harness the power of CMS while still utilizing the level-set method's ability to handle piecewise smooth manifolds and changing topologies. 
	
\subsection*{New Formulation}
	To harness the power of CMS, our new method locally initializes level-set like background nodes using constraints imposed by the tensor formulation of our surface. Then, using this local background node set, we can march our ambient solution in time either using the level-set equation or another solver from CMS ideas. Once we have marched a bit in time, we can recover our evolved surface by using the Coul-Newton method to get a new node band on our surface. Then, just like before, we can initialize a new local background set with constraints as needed.
	
	This new method differs from level-set methods in that each section of our curve or surface is evolved locally. This local formulation will hopefully lead to an algorithm that can take advantage of modern computer hardware to speed up computations. 
	
\subsection*{Important Equations}
	First, we would like to have a general representation of our surface which can be given
	\[
		\pos = \pos(S,t)
	\]
	where $ \pos $ is the position vector and $ \pos(S, t) $ is a vector valued function that takes surface coordinates and maps them to our ambient space. Next we can create the relationship between our surface coordinates and our ambient coordinates as
	\[
		Z^i = Z^i(S, t).
	\]
	Putting these together, we have
	\[
		\pos(S, t) = \pos(Z(s,t))
	\]
	which implies
	\[
		\frac{\pd \pos(S, t)}{\pd t} = \underbrace{\frac{\pd \pos}{\pd Z^i}}_{\mathbf{Z}_i} \underbrace{\frac{\pd Z^i(S, t)}{\pd t}}_{V^i}
	\]
	where $ \mathbf{Z}_i $ is our ambient covariant basis and $ V^i $ are the components of our coordinate velocity. Finally, we can get our tangential coordinate velocity by
	\[
		V^\alpha = V^i Z_i^\alpha
	\]
	where $ Z_i^\alpha $ is the shift tensor.
	
	One of the last key components to CMS is the invariant covariant time derivative which is given by
	\[
		\covt T = \frac{\pd T (S, t)}{\pd t} - V^\alpha \nabla_\alpha T
	\]
	Where $ T $ is an invariant. For variant tensors $ T_j^i $, we have
	\[
		\covt T_j^i = \frac{\pd T_j^i}{\pd t} - V^\gamma \nabla_\gamma T_j^i + V^k \Gamma_{km}^i T_j^m - V^k \Gamma_{kj}^m T_m^i
	\]
	where $ \Gamma_{ij}^k $ is the Christoffel Symbol of the first kind and is given by
	\[
		\Gamma_{ij}^k = \mathbf{Z}^k \cdot \frac{\partial \mathbf{Z}_i}{\partial Z^j}.
	\]
	
	Using these equations, we are able to rewrite our surface PDEs in a way that does not depend on the local surface parameterization giving us more freedom in how we evolve our surface.
	
\subsection*{Timeline}
	Our project has been underway for some time and has had many paths ranging from RBF generated finite surface differences on static surfaces to RBF based  level-set methods. With our future research focusing on the interface between level-set methods and point based methods, 
	
	\begin{enumerate}[label = \arabic*), itemsep = 0pt, topsep = 0pt]
		\item Flesh out potential criteria for initializing background node set.
		\item Generate error plots for test problems.
		\item Write paper and work on publishing (by the end of Spring 2022 semester).
		\item Attend Curves and Surfaces 2022 conference in France (June 20th - June 24)
	\end{enumerate}
\end{document}
