\documentclass[a4paper,11pt]{article}

\usepackage[utf8]{inputenc}
\usepackage[left=0.5in,right=0.5in,top=0.75in,bottom=0.75in]{geometry}
\usepackage{amsmath,amssymb,amsfonts}
\usepackage{pgfplots,graphicx,calc,changepage}
\pgfplotsset{compat=newest}
\usepackage{enumitem}
\usepackage{fancyhdr}
\usepackage[colorlinks = true, linkcolor = blue]{hyperref}

\DeclareMathOperator{\sech}{sech}
\DeclareMathOperator{\csch}{csch}
\DeclareMathOperator{\arcsec}{arcsec}
\DeclareMathOperator{\arccot}{arcCot}
\DeclareMathOperator{\arccsc}{arcCsc}
\DeclareMathOperator{\arccosh}{arcCosh}
\DeclareMathOperator{\arcsinh}{arcsinh}
\DeclareMathOperator{\arctanh}{arctanh}
\DeclareMathOperator{\arcsech}{arcsech}
\DeclareMathOperator{\arccsch}{arcCsch}
\DeclareMathOperator{\arccoth}{arcCoth} 
\newcommand{\nats}{\mathbb{N}}
\newcommand{\reals}{\mathbb{R}}
\newcommand{\rats}{\mathbb{Q}}
\newcommand{\ints}{\mathbb{Z}}
\newcommand{\pols}{\mathcal{P}}
\newcommand{\cants}{\Delta\!\!\!\!\Delta}
\newcommand{\eps}{\varepsilon}
\newcommand{\st}{\backepsilon}
\newcommand{\abs}[1]{\left| #1 \right|}
\newcommand{\dom}[1]{\mathrm{dom}\left(#1\right)}
\newcommand{\for}{\text{ for }}
\newcommand{\dd}[1]{\mathrm{d}#1}
\newcommand{\spn}{\mathrm{sp}}
\newcommand{\nul}{\mathcal{N}}
\newcommand{\col}{\mathrm{col}}
\newcommand{\rank}{\mathrm{rank}}
\newcommand{\norm}[1]{\lVert #1 \rVert}
\newcommand{\inner}[1]{\left\langle #1 \right\rangle}
\newcommand{\pmat}[1]{\begin{pmatrix} #1 \end{pmatrix}}
\renewcommand{\and}{\text{ and }}

\newsavebox{\qed}
\newenvironment{proof}[2][$\square$]
    {\setlength{\parskip}{0pt}\par\textit{Proof:} #2\setlength{\parskip}{0.25cm}
        \savebox{\qed}{#1}
        \begin{adjustwidth}{\widthof{Proof:}}{}
    }
    {
        \hfill\usebox{\qed}\end{adjustwidth}
    }

\pagestyle{fancy}
\fancyhead{}
\lhead{Caleb Jacobs}
\chead{APPM 5480: Asymptotics}
\rhead{Project Proposal}
\cfoot{}
\setlength{\headheight}{35pt}
\setlength{\parskip}{0.25cm}
\setlength{\parindent}{0pt}

\begin{document}
\begin{center}
	{\Large\textbf{RBF Interpolants Over Near-Flat Surfaces}}
\end{center}

\subsection*{Background and Problem}
Given a small parameter $ \eps $ that we will call the shape parameter and data $ \{\mathbf{x}_i, f_i\}_{i = 1}^n $, we can write out a radial basis function (RBF) based interpolant of the data as
\begin{equation}
	s(\mathbf{x}) = \sum_{i = 1}^n \lambda_i \phi_\eps(\norm{\mathbf{x} - \mathbf{x}_i}_2) \label{func:interp}
\end{equation}
where $ \lambda_i $ are interpolant weights and $ \phi_\eps $ is a radial function (such as $ \phi_\eps(r) = e^{-(\eps r)^2} $). In the most direct form, we can find $ \lambda_i $ by solving the system
\[
	\pmat{
		\phi_\eps (\norm{\mathbf{x}_1 - \mathbf{x}_1}) & \phi_\eps (\norm{\mathbf{x}_1 - \mathbf{x}_2}) & \cdots & \phi_\eps (\norm{\mathbf{x}_1 - \mathbf{x}_n}) \\
		\phi_\eps (\norm{\mathbf{x}_2 - \mathbf{x}_1}) & \phi_\eps (\norm{\mathbf{x}_2 - \mathbf{x}_2}) & \cdots & \phi_\eps (\norm{\mathbf{x}_2 - \mathbf{x}_n}) \\
		\vdots & \vdots & \ddots & \vdots \\
		\phi_\eps (\norm{\mathbf{x}_n - \mathbf{x}_1}) & \phi_\eps (\norm{\mathbf{x}_n - \mathbf{x}_2}) & \cdots & \phi_\eps (\norm{\mathbf{x}_n - \mathbf{x}_n})
	} \pmat{
		\lambda_1 \\ \lambda_2 \\ \vdots \\ \lambda_n
	} = \pmat{
		f_1 \\ f_2 \\ \vdots \\ f_n
	}.
\]
In solving this system numerically, it is known that taking $ \eps $ smaller and smaller tends to increase the accuracy of our interpolant. However, if $ \eps $ becomes too small, the matrix above becomes severely ill-conditioned and the interpolant becomes unusable numerically even when it should be well behaved. So, to find $ \lambda_i $ even when $ \eps $ is small, Bengt and some of his previous students have developed multiple methods that take $ \eps $ to be complex and then they perform a contour integral about $ \eps = 0 $ to recover the small $ \eps $ interpolant stably. The most recent and robust method, so called RBF-RA, is based on rational approximations of our underlying function. 

Now with small $ \eps $, we can apply RBF-RA to data that is on a surface with a large curvature $ \kappa $ or a completely flat surface (i.e. $ \kappa = 0 $) to get stable and desired results. However, applying RBF-RA to data on a surface that is nearly flat (i.e. $ 0 < \kappa \ll 1 $) leads to erroneous results. Both C\'ecile and Bengt have numerically explored this discrepancy with a few visuals (a singularity appears to pop up at the origin) but have not yet come to understand it nor fix it. Intuition tells us that this singularity for $ 0 < \kappa \ll 1 $ should be removable because both $ \kappa = 0 $ and $ 0 \ll \kappa $ lead to well behaved results. 

Herein lies my project, I am seeking to understand how our interpolant behaves when we perturb $ \kappa $ near zero.  The methods of Asymptotics appear to be a perfect candidate for developing this new understanding. However, before we can apply Asymptotics to this problem, we need to see where $ \kappa $ is embedded in our RBF interpolant. In this case, $ \kappa $ will appear in the data. For completely flat surfaces, the data all falls in a single plane. But, once we start perturbing $ \kappa $, our data will start to deviate from this plane. The simplest example of a surface with curvature $ \kappa $ is given by a circle of radius $ 1/\kappa $
\begin{equation}
	x^2 + y^2 = \frac{1}{\kappa^2} \implies y(x; \kappa) \label{equ:circ}
\end{equation}
Note that locally, any smooth curve can be approximated by this circle and that this circle can easily be extended to higher dimensions (i.e. hyperspheres). \textbf{So, our perturbation problem can be posed as asymptotically describe the interpolant $ s(x; \kappa) $ given in \eqref{func:interp} when we take our data to be $ \mathbf{x}_i = (x_i, y_i) $ subject to \eqref{equ:circ} with $ 0 < \kappa \ll 1 $}.

This problem is wide open and it doesn't appear anyone has been trying to tackle it in the field. Furthermore, if we can develop an understanding of how $ \kappa $ affects our interpolant, we could learn how to avoid numerical instability in the small $ \kappa $ regime and improve the RBF-RA method. As a personal impact of this project, C\'ecile, one of her grad students, and I have been using RBF-RA on general surfaces of which we are not guaranteed $ \kappa $ will be outside of the near-flat regime. So any new understanding of how $ \kappa $ affects RBF interpolants could greatly improve our results.
\end{document}
