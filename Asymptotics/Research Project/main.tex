\documentclass[a4paper,11pt]{article}

\usepackage[utf8]{inputenc}
\usepackage[left=0.5in,right=0.5in,top=0.6in,bottom=0.75in]{geometry}
\usepackage{amsmath,amssymb,amsfonts,bm}
\usepackage{pgfplots,graphicx,calc,changepage}
\pgfplotsset{compat=newest}
\usepackage{enumitem}
\usepackage{fancyhdr}
\usepackage[colorlinks = true, linkcolor = blue]{hyperref}

\DeclareMathOperator{\sech}{sech}
\DeclareMathOperator{\csch}{csch}
\DeclareMathOperator{\arcsec}{arcsec}
\DeclareMathOperator{\arccot}{arcCot}
\DeclareMathOperator{\arccsc}{arcCsc}
\DeclareMathOperator{\arccosh}{arcCosh}
\DeclareMathOperator{\arcsinh}{arcsinh}
\DeclareMathOperator{\arctanh}{arctanh}
\DeclareMathOperator{\arcsech}{arcsech}
\DeclareMathOperator{\arccsch}{arcCsch}
\DeclareMathOperator{\arccoth}{arcCoth} 
\newcommand{\nats}{\mathbb{N}}
\newcommand{\reals}{\mathbb{R}}
\newcommand{\rats}{\mathbb{Q}}
\newcommand{\ints}{\mathbb{Z}}
\newcommand{\pols}{\mathcal{P}}
\newcommand{\cants}{\Delta\!\!\!\!\Delta}
\newcommand{\eps}{\varepsilon}
\newcommand{\st}{\backepsilon}
\newcommand{\abs}[1]{\left| #1 \right|}
\newcommand{\dom}[1]{\mathrm{dom}\left(#1\right)}
\newcommand{\for}{\text{ for }}
\newcommand{\dd}[1]{\mathrm{d}#1}
\newcommand{\spn}{\mathrm{sp}}
\newcommand{\nul}{\mathcal{N}}
\newcommand{\col}{\mathrm{col}}
\newcommand{\rank}{\mathrm{rank}}
\newcommand{\norm}[1]{\lVert #1 \rVert}
\newcommand{\inner}[1]{\left\langle #1 \right\rangle}
\newcommand{\pmat}[1]{\begin{pmatrix} #1 \end{pmatrix}}
\renewcommand{\and}{\text{ and }}

\newsavebox{\qed}
\newenvironment{proof}[2][$\square$]
    {\setlength{\parskip}{0pt}\par\textit{Proof:} #2\setlength{\parskip}{0.25cm}
        \savebox{\qed}{#1}
        \begin{adjustwidth}{\widthof{Proof:}}{}
    }
    {
        \hfill\usebox{\qed}\end{adjustwidth}
    }

\pagestyle{fancy}
\fancyhead{}
\lhead{Caleb Jacobs}
\chead{APPM 5480: Asymptotics}
\rhead{Project Proposal}
\cfoot{}
\setlength{\headheight}{35pt}
\setlength{\parskip}{0.25cm}
\setlength{\parindent}{0pt}

\begin{document}
\begin{center}
	{\Large\textbf{RBF Interpolants Over Near-Flat Surfaces}}
\end{center}

\subsection*{Background and Problem}
Given a small parameter $ \eps $ that we will call the shape parameter and data $ \{\mathbf{x}_i, f_i\}_{i = 1}^n $, we can write out a radial basis function (RBF) based interpolant of the data as
\begin{equation}
	s(\mathbf{x}) = \sum_{i = 1}^n \lambda_i \phi_\eps(\norm{\mathbf{x} - \mathbf{x}_i}_2) \label{func:interp}
\end{equation}
where $ \lambda_i $ are interpolant weights and $ \phi_\eps $ is a radial function (in this case we will take $ \phi_\eps(r) = e^{-(\eps r)^2} $). In the most direct form, we can find $ \lambda_i $ by solving the system
\begin{equation}
	\underbrace{\pmat{
		\phi_\eps (\norm{\mathbf{x}_1 - \mathbf{x}_1}) & \phi_\eps (\norm{\mathbf{x}_1 - \mathbf{x}_2}) & \cdots & \phi_\eps (\norm{\mathbf{x}_1 - \mathbf{x}_n}) \\
		\phi_\eps (\norm{\mathbf{x}_2 - \mathbf{x}_1}) & \phi_\eps (\norm{\mathbf{x}_2 - \mathbf{x}_2}) & \cdots & \phi_\eps (\norm{\mathbf{x}_2 - \mathbf{x}_n}) \\
		\vdots & \vdots & \ddots & \vdots \\
		\phi_\eps (\norm{\mathbf{x}_n - \mathbf{x}_1}) & \phi_\eps (\norm{\mathbf{x}_n - \mathbf{x}_2}) & \cdots & \phi_\eps (\norm{\mathbf{x}_n - \mathbf{x}_n})
	}}_{A} \underbrace{\pmat{
		\lambda_1 \\ \lambda_2 \\ \vdots \\ \lambda_n
	}}_{\pmb{\lambda}} = \underbrace{\pmat{
		f_1 \\ f_2 \\ \vdots \\ f_n
	}}_{\pmb{F}}. \label{equ:direct}
\end{equation}
Where $ A $ is known as the collocation matrix. In solving this system numerically, it is known that taking $ \eps $ smaller and smaller tends to increase the accuracy of our interpolant. However, if $ \eps $ becomes too small, the matrix above becomes severely ill-conditioned and the interpolant becomes unusable numerically even when it should be well behaved. So, to find $ \lambda_i $ even when $ \eps $ is small, Bengt and some of his previous students have developed multiple methods that stably recover the small $ \eps $ interpolant. The most recent and robust method, so called RBF-RA, is based on rational approximations of our underlying function. 

Now with small $ \eps $, we can apply RBF-RA to data that is on a surface with a large curvature $ \kappa $ or a completely flat surface (i.e. $ \kappa = 0 $) to get stable, desired results. However, applying RBF-RA to data on a surface that is nearly flat (i.e. $ 0 < \kappa \ll 1 $) can lead to erroneous results. 

Herein lies my project, I am seeking to understand how our interpolation weights to \eqref{func:interp} behave when we perturb $ \kappa $ near zero. To explore this behavior, I will take nodes over a circle that just touches the origin with curvature $ \kappa $. Explicitly, we can define our $ n $ nodes as
\[
	\begin{cases}
		x_i = -1 + \dfrac{2}{n - 1} i, & i = 1, 2, \ldots, n \\
		y_i = \sqrt{\dfrac{1}{\kappa^2} - x_i^2} - \dfrac{1}{\kappa}, & 0 < \kappa \ll 1
	\end{cases}
\]
where our nodes are defined as the pairs $ \mathbf{x}_i = (x_i, y_i) $ for $ i = 1, 2, \ldots, n $. Then, if we plug these nodes into the collocation matrix above, we can expand $ A $ in $ \kappa $ about $ \kappa = 0 $ as
\begin{equation}
	A = A_0 + \kappa A_1 + \kappa^2 A_2 + \cdots. \label{pert:A}
\end{equation}
Inserting \eqref{pert:A} into \eqref{equ:direct} yields the perturbed system
\begin{equation}
	(A_0 + \kappa A_1 + \cdots) \pmb{\lambda} = \mathbf{F}. \label{equ:pert}
\end{equation}
which is a perfect candidate to be solved with asymptotic methods. Solving and understanding \eqref{equ:pert} is my main objective for this project but if time permits I would also like to explore the $ \kappa $-perturbed eigenvalue problem of $ A $,
\[
		(A_0 + \kappa A_1 + \cdots) \mathbf{x} = \lambda \mathbf{x}.
\]
 It will also be interesting to see if other small curvature node sets behave differently than the proposed node set above (this could also include higher dimensional node sets). In any case, I will need to take a closer look at our collocation matrix to see if asymptotic order is broken for $ \eps $ small enough which might lead to singularly perturbed solutions.
\end{document}
